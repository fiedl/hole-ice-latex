% Hyperlinks und Kreuzverweise zu anderen Dokumenten
\usepackage{url}
% %\usepackage{xr}
% %\usepackage{xr-hyper}

\usepackage{footmisc}

\usepackage{xcolor}
\definecolor{bluelink}{rgb}{.247,.443,.729}
\definecolor{greenlink}{rgb}{.23,.57,.23}
\definecolor{redlink}{rgb}{.51,.169,.129}

\newcommand\defaulthypersetup{\hypersetup{allcolors=black, citecolor=greenlink, linkcolor=bluelink, urlcolor=black}}
%\newcommand\defaulthypersetup{\hypersetup{allcolors=black, citecolor=greenlink, linkcolor=redlink, urlcolor=bluelink}}

\usepackage[breaklinks, colorlinks]{hyperref}\defaulthypersetup
\usepackage[hyphenbreaks]{breakurl}

% https://tex.stackexchange.com/a/50754/70789
% % \hypersetup{
% %   colorlinks = false,
% %   linkbordercolor = {white},
% % }

% https://tex.stackexchange.com/a/668/70789
\urlstyle{sf}

\newcommand\docframe[1]{%
  \vspace{1em}\fbox{\minipage{\textwidth}#1\endminipage}\vspace{1em}%
}

\newcommand\iconpar[2]{%
  \docframe{\iconparwithoutframe{#1}{#2}}%
}

\newcommand\iconparwithoutframe[2]{%
  \begin{tabularx}{\textwidth}{lL}%
    \adjustbox{minipage=6mm, center=6mm, valign=t, raise=-2mm}{\center{#1}} & \footnotesize\sffamily #2%
  \end{tabularx}%
}

% \newcommand\githubicon{\faGithub}
% \newcommand\githubpar[1]{\iconpar{\githubicon}{#1}}
%
% \newcommand\webicon{\hspace*{1mm}\faExternalLink}
% \newcommand\webpar[1]{\iconpar{\webicon}{#1}}

\newcommand\documentationicon{\faBook}
\newcommand\docpar[1]{\iconpar{\documentationicon}{#1}}
\newcommand\docparwithoutframe[1]{\iconparwithoutframe{\documentationicon}{#1}}

\newcommand\sourcecodeicon{\faFileTextO}
\newcommand\sourcepar[1]{\iconpar{\sourcecodeicon}{#1}}
\newcommand\sourceparwithoutframe[1]{\iconparwithoutframe{\sourcecodeicon}{#1}}

\newcommand\youtubeicon{\faVideoCamera} %\faYoutube \faVideoFileO
\newcommand\youtubepar[1]{\iconpar{\youtubeicon}{#1}}