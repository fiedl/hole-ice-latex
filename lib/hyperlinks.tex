% Hyperlinks und Kreuzverweise zu anderen Dokumenten
\usepackage{url}
% %\usepackage{xr}
% %\usepackage{xr-hyper}

\usepackage{xcolor}
\definecolor{citelinkcolor}{rgb}{.23,.57,.23}

\usepackage[breaklinks, colorlinks, allcolors=black, citecolor=citelinkcolor]{hyperref}
\usepackage[hyphenbreaks]{breakurl}

% https://tex.stackexchange.com/a/50754/70789
% % \hypersetup{
% %   colorlinks = false,
% %   linkbordercolor = {white},
% % }

% https://tex.stackexchange.com/a/668/70789
\urlstyle{sf}

\newcommand\iconpar[2]{%
  \vspace{1em}\fbox{%
    \begin{tabularx}{\textwidth}{lL}%
      \adjustbox{minipage=6mm, center=6mm, valign=t, raise=-2mm}{\center{#1}} & \footnotesize\sffamily #2%
    \end{tabularx}%
  }\vspace{1em}%
}

\newcommand\githubicon{\faGithub}
\newcommand\githubpar[1]{\iconpar{\githubicon}{#1}}

\newcommand\webicon{\hspace*{1mm}\faExternalLink}
\newcommand\webpar[1]{\iconpar{\webicon}{#1}}

\newcommand\documentationicon{\faBook}
\newcommand\docpar[1]{\iconpar{\documentationicon}{#1}}

\newcommand\sourcecodeicon{\faFileTextO}
\newcommand\sourcepar[1]{\iconpar{\sourcecodeicon}{#1}}

\newcommand\youtubeicon{\faVideoCamera} %\faYoutube \faVideoFileO
\newcommand\youtubepar[1]{\iconpar{\youtubeicon}{#1}}