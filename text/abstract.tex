\cleardoublepage
\thispagestyle{empty}

\begin{abstract}

\icecube is a neutrino observatory at Earth's South Pole that uses glacial ice as detector medium. Secondary particles from neutrino interactions produce Cherenkov light, which is detected by an array of photo detectors deployed within the ice. In distinction from the glacial bulk ice, hole ice is the refrozen water in the drill holes around the detector modules, and is expected to have different optical properties than the bulk ice.
Aiming to improve detector precision, this study presents a new method to simulate the propagation of light through the hole ice, introducing several new calibration parameters. The validity of the method is supported by a series of statistical cross checks, and by comparison to measurement and simulation results from other calibration studies.
Evaluating calibration data indicates a strongly asymmetric shielding of the detector modules. A preliminary analysis suggests that this cannot be accounted for by the shadow of cables, but can be explained by hole ice with a suitable scattering length, size, and position relative to the detector modules.
The hole-ice approximation, which is used in the standard simulation chain is found to disagree with all existing direct-propagation methods and should be recalculated with a new direct-simulation run.

\end{abstract}
