%!TEX TS-program = ../make.zsh

\subsection{Unit Tests and Consistency Checks}
\label{sec:unit_tests_and_cross_checks}

\todo{Paragraph about several kinds of tests}
\todo{Besser gliedern: Es gibt (a) interne Konsistenz-Checks in clsim, (b) Unit-Tests, (c) statistische Konsistenz-Checks}

- single photons, unit tests
- several photons behaving the same way, instant absorption
- sample several photons, statistical properties, distribution of observables like arrival time or path length, qualitatively, quantitatively

### Unit Tests With Single Photons
\label{sec:unit_tests}

In order to verify that individual components (``units'') of the implementation perform their tasks as expected, that is to say produce results as expected, unit tests were implemented for the algorithm that calculates intersections as well as for the hole-ice-correction algorithm.

\sourcepar{The unit tests for the intersection algorithm can be found at \url{https://github.com/fiedl/clsim/blob/sf/hole-ice-2017/resources/kernels/lib/intersection/intersection_test.c}, the unit tests for the hole-ice-correction algorithm at \url{https://github.com/fiedl/clsim/blob/sf/hole-ice-2017/resources/kernels/lib/hole_ice/hole_ice_test.c}.}

\paragraph{Task} The task of unit tests is to test individual components of a software. In this case, the tests execute separate functions of the new algorithms with fixed input parameters and check whether the functions return the expected results that, in preparation of the test, have been obtained by other means, either by a separete program, using a separate programming language, a separate algorithm, or via calculations by hand.

In this study, unit tests implement single photons that cross hole-ice cylinders in a pre-defined way and check whether the intersection algorithm determines the correct intersection points, whether the 2d-3d projections are handled correctly, and whether the hole-ice-correction algorithm determines the expected corrections for the geometric distances according to the hole-ice parameters provided by the test scenario.

Extreme examples can be designed to produce simple results that can be calculated by hand and verified by intuition. For example, for hole-ice cylinders configured for instant absorption, the absorption point is identical to the intersection point of the photon trajectory and the cylinder.

More complex examples require more involved calculations by hand or using other software like \noun{Python} scripts or specialized tools like \noun{GeoGebra}\footnote{GeoGebra mathematical tools, \url{https://www.geogebra.org}} (see figure \ref{fig:Uchie2aa}) for calculating intersections.
% https://github.com/fiedl/hole-ice-study/issues/20
% https://github.com/fiedl/hole-ice-study/issues/25

% \begin{figure}[htbp]
%   \image{geogebra}
%   \caption{A specialized software tool like \noun{GeoGebra} can be used to obtain or verify individial calculation results, which then are used in unit tests to verify the results of the implemented algorithms.}
%   \label{fig:Uchie2aa}
% \end{figure}

\paragraph{Testing Framework}
This study uses the \noun{gtest}\footnote{Google Test Framework, gtest, \url{https://github.com/google/googletest}} testing framework.

% Notes: 2017-07-21
This framework has been chosen due to its good documentation, wide adoption and slim architecture that made it possible to use the framework to test individual components of the new source code without interfering with the rest of the icesim framwork.

\docpar{The introductory documentation of gtest can be found at \url{https://github.com/google/googletest/blob/master/googletest/docs/primer.md}.}

\paragraph{Pros and Cons}
Unit tests are most useful to ensure stability when adjusting, refactoring or rewriting software components. Even after replacing a large source code, unit tests can make sure that the software still produces the same results as before.

Tests also allow for so-called test-driven development where the expected results are specified first, and then the software is built or changed iteratively until it produces the expected results.

This technique works best when the components to be tested have small interfaces because the technique requires the tests to provide all input paramters for the components to be tested.

Unexpected issues may arise when unit tests are run on another architecture as the production code. For example, the unit tests used in this study were insensitive to certain driver issues and numerical issues\footnote{\mref{technical-issues section}} that did only occur when running the software on the GPUs of the computing cluster and did not occur when running the unit tests on the local CPU of the development machine.

Unit tests are, by design, also insensitive to high-level problems. Each individual component may produce the results expected from this component while still issues may arise when components are not tied together correctly, or because the expectations for an individual component may be wrong, which becomes apparent only when adopting a larger perspective rather than the focused perspective on individual components.

To increase confidence in the new software, therefore, in addition to unit testing, high-level consistency checks need to be performed as described in the following subsections.


### Instant-Absorption Tests

A simple high-level test can be performed by starting photons towards a cylinder, which is configured for instant absorption. After propagating the photons using the new media-propagation algorithm and recording the path of each individual photon, one can verify the results using a python script that makes sure than no recorded photon position is inside the instant-absorption cylinder.

Additionaly, one can visualize the simulation using the \noun{Steamshovel} event viewer to verify that no photon enters the instant-absorption cylinder as shown in figure \ref{fig:moo9Eiqu}.

% instant absorption: issue #22

\begin{figure}[htbp]
  \subcaptionbox{Starting from different positions into the same direction. View from above.}{\includegraphics[width=0.49\textwidth]{img/instant-absorption-steamshovel-moo9Eiqu}}\hfill
  \subcaptionbox{Starting from the same position into different directions.}{\includegraphics[width=.49\textwidth, trim={0 0 40mm 0}, clip]{img/instant-absorption-steamshovel-Zae4phei}}
  \caption{Visualizing an instant-absorption test using the \noun{Steamshovel} event viewer. In the simulation, photons are started towards a cylinder configured for instant absorption. If the medium-propagation algorithm works as expected, no photon can get inside the cylinder.}
  \label{fig:moo9Eiqu}
\end{figure}

A related, but more complex scenario is starting photons within two nested cylinders where the inner cylinder is configured for a short scattering length and the outer cylinder is configured for instant absorption (figure \ref{fig:sahmoo8O}).

% https://github.com/fiedl/hole-ice-study/issues/47

\begin{figure}[htbp]
  \subcaptionbox{View from above.}{\includegraphics[width=.32\textwidth, trim={0 3cm 16cm 0}, clip]{img/instant-absorption-steamshovel-sahmoo8O-above}}\hfill
  \subcaptionbox{View from the side.}{\includegraphics[width=.32\textwidth, trim={6cm 4cm 13cm 1.5cm}, clip]{img/instant-absorption-steamshovel-sahmoo8O-3d}}\hfill
  \subcaptionbox{Instant absorption of the outer cylinder turned off.}{\includegraphics[width=.32\textwidth, trim={6cm 4cm 13cm 1.5cm}, clip]{img/instant-absorption-steamshovel-sahmoo8O-turned-off}}
  \caption{Visualizing an instant-absorption test where photons are started within two nested cylinders. The outer cylinder is configured for instant absorption. No photon can pass through the area between both cylinders unless the instant absorption is turned off.}
  \label{fig:sahmoo8O}
\end{figure}


### Arrival-Time Distributions

% https://github.com/fiedl/hole-ice-study/issues/91

One way to test the behaviour of statistical properties of a sample of photons is to plot the arrival-time distribution of photons travelling from a central position to receiving optical modules around the starting position.

\sourcepar{The source for the simulations and for creating these histograms can be found in \issue{91}.}

\begin{figure}[htbp]
  \subcaptionbox{Top view of the detector strings in this simulation. The photons are started at the middle optical module of string 63, and are received by the optical modules of the surrounding strings 70, 71, 64, 55, 54, and 62.}{\includegraphics[width=.48\linewidth]{img/flasher-scenario}}
  \hfill
  \subcaptionbox{For different hole-ice configurations, the simulation results in different photon-arrival-time distributions. The dashed lines indicate the mean arrival times. With stronger hole-ice effects, less photons arrive in total, the photons arrive later, and more distributed.}{\includegraphics[width=.48\linewidth]{img/arrival-time-distribution-eipau6Ag}}
  \caption{Testing the arrival-time distribution of photons propagating from a central position to receiving optical modules around the starting position. The simulation includes hole-ice cylinders around the detector strings and is performed for several configurations to observe different strengths of the hole-ice effect.}
  \label{fig:eipau6Ag}
\end{figure}

Figure \ref{fig:eipau6Ag} shows arrival-time distributions for this scenario being carried out with a flasher experiment compared to two simulations with different hole-ice configurations each.

% arrival-time distribution #91, estimation see 2018-07-18.

An estimation based on the mean scattering angle and the radius of the hole-ice cylinder suggests that one would need a difference of the hole-ice cylinder's radius of several meters to cause a difference in the mean arrival time in the order of $100\ns$. While this is no realistic scenario as the drill hole's radius is only about $50\cm$, the simulation allows to use extreme scenarios to observe the effects more noticeable.

From the comparison of the arrival-time distributions, note that the distribution that is based on real data is comparable to the ones based on simulations, which suggests that the new algorithm does not introduced unexpected, unphysical effects to the photon propagation.

\todo{Check whether the simulation actually sets the time of the photon correctly in each simulation step.}

In the simulation where the hole-ice cylinders have a much larger radius, the effects of the hole ice should be stronger. Indeed, the comparison of the distributions show three effects to expect from this assumption:

First, in the simulation with stronger hole-ice effect, less photons arrive at the receiving optical modules in total as more photons are scattered away by the hole ice.

Second, for stronger hole ice, the photons arrive later at the receiving optical modules, because they spend more time scattering randomly within the hole-ice cylinder before reaching the receiving optical module.

Third, for stronger hole ice, the left-hand side of the distribution histogram is less steep, because with more scattering points on the photon's path, there is a larger number of possible paths, which leads to the arrival time being more distributed.

While these effects confirm the expectations qualitatively, examining other high-level observables allows to compare expectations to the simulations' results even quantitatively. This is the subject of the following sections, beginning with observing the photons' path-length distributions.


\subsubsection{Exponential Distribution of the Total Path Length}

% https://github.com/fiedl/hole-ice-study/issues/64

Let the \textbf{Total path length} of a photon be the summed distance from the position where the photon is created, along the photon's path, to the position where the photon is absorbed.

The total path length $X$ of photons that propagate in a medium with an absorption probability that is the same everywhere in the medium, for example within a confined area within the bulk ice, is expected to follow an exponential distribution with a rate parameter corresponding to the absorption probability, or equivalently the absorption length $\lambda$ within the medium.\footnote{If the photons do not depend on each other, the nuber of photons that are absorbed at a certain path length does only depend on the absorption probability, or equivalently on the absorption length $\lambda$, and the number of photons remaining at this path length. When a photon is absorbed, the number of remaining photons decreases. Thus, the change in the number of photons is proportional to the number of photons, causing the number of photons absorbed at a certain path length following an exponential distribution.}

\begin{equation}
  X: \ \ P(X \leq x) \propto e^{- \sfrac{x}{\lambda}}, \ \ x \in \reals^+_0
\end{equation}

When propagating photons within a hole-ice cylinder, their path length should also follow an exponential distribution, but with adifferent absorption length $\lambda\hi$, as long as the absorption length is sufficiently small such that most photons won't leave the cylinder before being absorbed.

This behaviour can be verified using a simulation. % The scattering length may be chosen arbitrarily small to make it unlikely that the photon will leave the hole-ice cylinder before it is absorbed.

\begin{figure}[htb]
  \image{cross-check-64-exponential-distribution.png}
  \caption{Distribution of the total path length of simulated photons both started and absorbed within a hole-ice cylinder using the new medium-propagation algorithm. As expected, the distribution follows an exponential curve. The fitted absorption length $\lambda_\text{abs}=0.1003\m \pm 0.0011\m$ is in agreement with the true absorption length of $0.1000\m$ used in the simulation.}
\end{figure}

\paragraph{Simulation scenario} In the simulation, a pencil beam of $10^4$ photons is started within a hole-ice cylinder with a radius of $1.0\m$. The photons are started with a distance of $0.9\m$ to the cylinder center towards the cylinder center. Let the effective scattering length within the cylinder be $100.0\m$ and the absorption length be $0.1\m$.

\sourcepar{The implementation of this consistency check can be found in \issue{64}.}

\todo{Number of significant digits}

% [2018-05-14 16:33:11] fiedl@fiedl-mbp ~/hole-ice-study/scripts/FiringRange master ⚡
% ▶ rm tmp/gcd_with_hole_ice.i3
% ./run.rb \
%     --hole-ice-radius=1.00 \
%     --effective-scattering-length=100.0 \
%     --absorption-length=0.10 \
%     --distance=0.90 \
%     --number-of-photons=10000 \
%     --cpu --save-photon-paths \
%     --number-of-runs=1 --number-of-parallel-runs=1 \
%     |grep "CROSS CHECK" |grep -v "INFO" \
%     > ~/hole-ice-study/results/cross_checks/cross_check_64.txt
% steamshovel tmp/propagated_photons.i3

\begin{figure}
  \image{cross-check-64-steamshovel}
  \caption{Viewing this cross check in steamshovel. The DOM is shown only to illustrate the size of the scenary and is configured not to interact with the simulated photons.}
\end{figure}

\paragraph{Observable} For each simulated photon, record the total path length, and plot the distribution of the path lengths.

\paragraph{Expectation}

- The probability of absorption is the same at each point within the hole-ice cylinder
- the distribution of the total path length should folow an exponential curve governed by the photon absorption length within the hole-ice cylinder.
- Let $N$ be the total number of started photons in the simulation. Let $\lambdaabs$ be the photon absorption length within the hole-ice cylinder.
- The probability $p(x):=f\,\dx$ for a photon to be absorbed within its path length interval $[x; x + \dx[$ is the same as long as the photon stays within the hole ice.
- Let $n(x)$ be the number of photons with a path length of $x$ or more, i.e. the number of photons that still exist after travelling the distance $x$. \footnote{\todo{"after" ist nicht präzise}}
- The number of photons that are absorbed within the interval $[x; x + \dx[$ should (in the limit of many photons) be $-\dn(x) := p(x)\ n(x) = f\,\dx\,n(x)$.

$$ \frac{\d}{\dx}\ n(x) = -f\ n(x) $$

- As the derivative of $n$ is proportional to $n$, $n$ is an exponential: $n(x) = a\,\e^{b\,x}$

$$ \frac{\d}{\dx}\ n(x)
  = \frac{\d}{\dx}\ a\,\e^{b\,x}
  = a\,b\,\e^{b\,x}
  = -f\ n(x)
  = -f\,a\,\e^{b\,x} $$

- Therefore, $b = -f$.
- As the absorption length $\lambdaabs$ is defined as the distance after that the number of photons has dropped to $1/\e$ of the original number: $b = -1/\lambdaabs$.

% absorption probability p at each point.
% p = dx f
% probability to be absorbed after distance x: P(x) is proportional to number of pho
% the number of photon with a total path length of x is proportional to the total number of photons and the probability of a photon to have this path length
% probability of a photon having a path length of x: depends on the absorption length or the absorption probability at each point
% \int_0^x{dx f}

$$
  n(x) = n_0 \cdot \e^{-\frac{x}{\lambda_\text{abs}}}
$$

% Falsch [n(x) ist die Anzahl von Photonen, die noch da sind.]: The number of photons $n(x)$ with total path length $x$, $\lambda_\text{abs}$ is the absorption length within the hole-ice cylinder. $N$ is the total number of photons started in the simulation.
%
% $$
%   n(x) \propto N \cdot \e^{-\frac{x}{\lambda_\text{abs}}}
% $$

- In a histogram of the total path lengths $x$, the bin height is proportional to the number of photons that are absorbed within the interval $[x; x + \dx[$, which is $-\dn(x)$.

$$ \frac{\d}{\dx}\ n(x) = -f\ n(x)
  = -f\ a\,\e^{-f\,x} $$

- Therefore, the bin height $b(x)$ of the histogram should also follow an exponential curve:

$$ b(x) \propto \dn(x) \propto \e^{-x/\lambdaabs} $$

- And one should be able to determine the absorption length $\lambdaabs$ by fitting the histogram bins to an exponential function.

\paragraph{Confirmation} Indeed, the simulation yields the expected distribution of the total path length. Via a curve fit, the absorption length $\lambdaabs$ could be determined for this simulation to be $\lambdaabs = 0.1003\m \pm 0.0011\m$, which is in accordance with the true absorption length of $0.1000\m$ used in the simulation.


\subsubsection{Piece-wise exponential distribution of the total path length for one medium boundary}

% https://github.com/fiedl/hole-ice-study/issues/65

This consistency check confirms that the medium boundary is handled correctly when photons enter a hole-ice cylinder from outside.

% // Ice properties outside the hole ice:
% floating_t local_scattering_lengths[MEDIUM_LAYERS] = {1000000.0};
% floating_t local_absorption_lengths[MEDIUM_LAYERS] = {1.0};

% [2018-05-14 17:56:59] fiedl@fiedl-mbp ~/hole-ice-study/scripts/FiringRange master ⚡
%
% # Ice properties inside the hole ice:
% rm tmp/gcd_with_hole_ice.i3
% ./run.rb \
%     --hole-ice-radius=1.00 \
%     --effective-scattering-length=100.0 \
%     --absorption-length=0.10 \
%     --distance=2.0 \
%     --number-of-photons=100000 \
%     --cpu --save-photon-paths \
%     --number-of-runs=1 --number-of-parallel-runs=1 \
%     |grep "CROSS CHECK" |grep -v "INFO" \
%     > ~/hole-ice-study/results/cross_checks/cross_check_65.txt

\paragraph{Simulation scenario} Start a pencil beam of $10^5$ photons in a distance of $2.0\m$ to cylinder center towards the hole-ice cylinder with a radius of $1.0\m$. Set the effective scattering length outside to $10^6\m$, inside to $100\m$. Set the absorption length outside to $1.0\m$, inside to $0.10\m$.\footnote{This simulation is implemented and documented at \url{https://github.com/fiedl/hole-ice-study/issues/65}.}

\begin{figure}
  \image{cross-check-65-steamshovel}
  \caption{Viewing the simulation in steamshovel: The photons are started outside the hole-ice cylinder on the left hand side. As the scattering length within the cylinder is smaller, the photons scatter a lot more within the cylinder. The DOM is shown only to illustrate the size of the scenary and is configured not to interact with the simulated photons.}
\end{figure}

\paragraph{Observable} As before, for each simulated photon, record the total path length, i.e. the total travelled distance from the starting point of the photon to the point where the photon is absorbed. Plot the distribution of these total path lengths.

\paragraph{Expectation} As the absorption lengths are different outside and inside the cylinder, the histogram should now follow two separate exponential curves: The left hand side of the histogram, which corresponds to the area outside the cylinder, should follow an exponential curve governed by the absorption length outside the cylinder. The right hand side, which corresponds to the area within the cylinder, should follow an exponential curve goverened by the absorption length within the hole-ice cyliner.

When plotting the number $n(x)$ of photons that still exist after travelling a path length of $x$, $n(x)$ should be a piece-wise defined function consisting of two exponential curves, but in contrast to the histogram of the path lengths, be continous at the medium border.

\todo{Integral-Formel aus Notizen vom 2018-06-03 prüfen und übertragen}

\begin{figure}
  \image{cross-check-65-histogram}
  \caption{Distribution of the total path length of simulated photons both started outside and absorbed within a hole-ice cylinder using the new medium-propagation algorithm. As expected, the distribution follows two exponential curves. The fitted absorption lengths $\lambda_\text{abs,1} = 1.0161\m \pm 0.3099\m$ and $\lambda_\text{abs,2} = 0.1012\m \pm 0.0144\m$ are in agreement with the true absorption length outside of $1.0000\m$ and within the hole-ice cylinder of $0.1000\m$ set in the simulation.}
\end{figure}

\todo{Plot des negativ-kumulativen Histogramms}

\paragraph{Confirmation} The simulation yields the expected distribution of the total path length. Via a curve fit, the absorption lengths $\lambda_\text{abs,1} = 1.0161\m \pm 0.3099\m$ and $\lambda_\text{abs,2} = 0.1012\m \pm 0.0144\m$ are in determined in accordance with the true absorption length outside of $1.0000\m$ and within the hole-ice cylinder of $0.1000\m$ set in the simulation.


\subsubsection{Piece-wise exponential distribution of the total path length for two medium boundaries}

% https://github.com/fiedl/hole-ice-study/issues/66

This consistency check confirms that the medium boundary is handled correctly when photons leave a hole-ice cylinder.

\paragraph{Simulation scenario} Start a pencil beam of $10^5$ photons in a distance of $1.5\m$ to the cylinder center towards the hole-ice cylinder with a radius of $0.5\m$. Set the effective scattering length outside and inside to $10^6\m$. Set the absorption length outside to $1.0\m$, inside to $0.75\m$.\footnote{This simulation is implemented and documented at \url{https://github.com/fiedl/hole-ice-study/issues/66}.}

\begin{figure}
  \image{cross-check-66-steamshovel}
  \caption{Viewing the simulation in steamshovel: The photons are started outside the hole-ice cylinder on the left hand side. Some of them are absorbed before entering the cylinder, some within the cylinder and some after leaving the cylinder on the right hand side. The DOM is shown only to illustrate the size of the scenary and is configured not to interact with the simulated photons.}
\end{figure}

\paragraph{Observable} As before, for each simulated photon, record the total path length, i.e. the total travelled distance from the starting point of the photon to the point where the photon is absorbed. Plot the distribution of these total path lengths.

\paragraph{Expectation} As the simulated photons may now cross two different medium boundaries, the distribution of the path lengths should now follow three exponential curves: On the left hand side of the histogram, which corresponds to the photons that are absorbed before entering the hole ice, the histogram should follow an exponential curve governed by the absorption length outside. In the middle, which corresponds to the photons that are absorbed inside the cylinder, the histogram should follow an exponential curve governed by the absorption length within the hole ice. On the right hand side, which corresponds to the photons that are absorbed after leaving the hole-ice cylinder, the histogram should follow an exponential curve, again, governed by the absorption length outside the cylinder.

\begin{figure}
  \image{cross-check-66-histogram}
  \caption{Distribution of the total path length of simulated photons, which start outside the cylinder, and may be absorbed before, within or after passing the cylinder, using the new medium-propagation algorithm. As expected, the distribution follows three exponential curves. The fitted absorption lengths $\lambda_\text{abs,1} = 1.0213\m \pm 0.2227\m$, $\lambda_\text{abs,2} = 0.6845\m \pm 0.1888\m$ and $\lambda_\text{abs,3} = 1.0722\m \pm 0.2343\m$ are in agreement with the true absorption length outside of $1.0000\m$ and within the hole-ice cylinder of $0.7500\m$ set in the simulation.}
\end{figure}

\todo{Plot des negativ-kumulativen Histogramms}

\paragraph{Confirmation} The simulation yields the expected distribution of the total path length. Via a curve fit, the absorption lengths $\lambda_\text{abs,1} = 1.0213\m \pm 0.2227\m$, $\lambda_\text{abs,2} = 0.6845\m \pm 0.1888\m$ and $\lambda_\text{abs,3} = 1.0722\m \pm 0.2343\m$ are in determined in accordance with the true absorption length outside of $1.0000\m$ and within the hole-ice cylinder of $0.7500\m$ set in the simulation.

### Absorption Length Distributions

- cross checks #63
- exponentieller abfall
  - without border #64
  - with 1 border #65
  - with 2 borders #66
- zylinder-übergang #47
- distance to hole-ice center #67
- distance to next scattering point #71
- scattering and absorption #72 (?)




