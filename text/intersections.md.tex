%!TEX root = ../diplomarbeit.tex

\subsection{Calculating Intersections of Photon Trajectories With Hole-Ice Cylinders}
\label{sec:intersections}

### Analytic Approach

\image{intersection-Kahm4UeY}

Consider the starting point $A := (A_x, A_y)$ and the ending point $B := (B_x, B_y)$ of the trajectory.

The length $\len{AB}$ of the trajectory is given by the norm $\norm{.}$ of the vector difference $\vec{AB}$ of starting and ending point.
\begin{equation} \notag
  \len{AB} = \norm{\vec{AB}}, \ \ \vec{AB} \identical \vec{B} - \vec{A}, \ \ \norm{\vec{v}} \identical \sqrt{v_x^2 + v_y^2}
\end{equation}

In order to find the intersection points $X_1$ and $X_2$, solve the vectorial system of equations

\begin{equation}
  \vec{A} + s \, \vec{AB} = \vec{M} + \vec{MX}
\end{equation}
\begin{equation}
  \norm{\vec{MX}} = r
\end{equation}

for the scalar parameter $s$. The equation system is quadratic in $s$ resulting in zero, one or two solutions.

\begin{equation}
  s_{1,2} = \frac{-\beta \mp \sqrt{\beta^2 - 4\alpha\gamma}}{2\alpha}
\end{equation}
\begin{equation}
  \alpha = (B_x - A_x)^2 + (B_y - A_y)^2
\end{equation}
\begin{equation}
  \beta = 2\,A_x(B_x-A_x) + 2\,A_y(B_y-Ay) - 2\,M_x(B_x-A_x) - 2\,M_y(B_y-A_y)
\end{equation}
\begin{equation}
  \gamma = A_x^2 - 2\,A_x\,M_x^2 + M_x^2 + A_y^2 - 2\,A_y\,M_y + M_y^2 - r^2
\end{equation}

The term under the square root is also called \textbf{discriminant} $D$.
\begin{equation}
  D = \beta^2 - 4\alpha\gamma
\end{equation}
For $D < 0$, the square root does not exist in $\reals$ and therefore, no intersection point exists. For $D = 0$, there is only one intersection point, which is a tangent point. For $D > 0$, there are two intersection points.

Note that $s = 0$ at the starting point $A$, $s = 1$ at the ending point $B$, $s = s_1$ at the first intersection point $X_1$ and $s = s_2$ at the second intersection point $X_2$.

Therefore, the intersection point coordinate vectors $\vec{X_1}$ and $\vec{X_2}$ are given by:
\begin{equation}
  \vec{X_1} = \vec{A} + s_1 \, \vec{AB}
\end{equation}
\begin{equation}
  \vec{X_2} = \vec{A} + s_2 \, \vec{AB}
\end{equation}

This result can be verified using a symbolic mathematics library like \noun{SymPy}\footnote{\textbf{SymPy} is a Python library for symbolic mathematics. \url{http://sympy.org}}.

% diplomarbeit/Logbuch/2018-02-13_schnittpunktberechnung.py

\begin{python}
# python

from sympy import *
init_printing(use_unicode=True)

# Variables
# ---------

Ax, Ay = symbols("Ax Ay")
Bx, By = symbols("Bx By")
Mx, My = symbols("Mx My")
Px, Py = symbols("Px Py")
r = symbols("r")
s = symbols("s")

vx = Bx - Ax
vy = By - Ay

# First ansatz
# ------------

lhs = (Ax + s * vx - Mx)**2 + (Ay + s * vy - My)**2 - r**2
expanded_lhs = expand(lhs)

intersection_equation = Eq(lhs, 0)

solution = solve(intersection_equation, s)
s1 = solution[0]
s2 = solution[1]

# Second ansatz
# -------------

alpha = (By - Ay)**2 + (Bx - Ax)**2
beta = 2 * Ay * (By - Ay) + 2 * Ax * (Bx - Ax) - 2 * My * (By - Ay) - 2 * Mx * (Bx - Ax)
gamma = Ay**2 - 2 * Ay * My + My**2 - r**2 + Ax**2 - 2 * Ax * Mx + Mx**2
discriminant = (beta**2 - 4 * alpha * gamma)

ss1 = (- beta - sqrt(beta**2 - 4 * alpha * gamma)) / (2 * alpha)
ss2 = (- beta + sqrt(beta**2 - 4 * alpha * gamma)) / (2 * alpha)

# Compare ansatzes
# ----------------

# See also: http://docs.sympy.org/latest/tutorial/gotchas.html

# # Does not work, because `==` checks object (structural) equality
# s1 == ss1
# s1 == ss2
# s2 == ss1
# s2 == ss2

# Should be 0 for the matching pairs:
simplify(s1 - ss1)  # => 0
simplify(s1 - ss2)
simplify(s2 - ss1)
simplify(s2 - ss2)  # => 0

# Or numerically:
s1.equals(ss1)  # True
s2.equals(ss2)  # True
\end{python}



### Geometric Approach

It turns out, the simulation on the GPU is faster and more precise if one does not treat the coordinates, e.g. $A_x, A_y$ as separate quantities but keeps the calculations as much as possible vectorial and uses the GPU-native vector operation functions like `dot` for the vector scalar product ("dot product").

\image{intersection-vectorial-auloLai2}

\paragraph{Goal} The goal is to find the distances $\len{AX_1}, \len{AX_2}$ from the starting point $A$ to the intersection points $X_1, X_2$.

\paragraph{Relations}

Consider a projection of $M$ onto $AB$. The projected point is called $M'$.

Given $\len{A M'}$ and $\len{X_1 M'}$, the distances to the intersection points can be calculated as:

\begin{equation} \len{A X_{1,2}} = \len{A M'} \mp \len{X_1 M'} \end{equation}

Given $\vec{AM}$ and the unit vector $\vec{v}$ in $\vec{AB}$ direction, $\len{AM'}$ can be calculated using a vector projection:

\begin{equation} \len{AM'} = \vec{AM} \cdot \vec{v} \end{equation}

The operator $\cdot$ is the scalar product (dot product). The unit vector $\vec{v}$ in $\vec{AB}$ direction is $\vec{v} = \vec{AB} / \len{AB}$.

The distance $m:=\len{MM'}$ on right hand side of the triangle $\triangle AM'M$ can be calculated using the pythagorean theorem:

\begin{equation} m: \len{AM'}^2 + m^2 = \len{AM}^2 \end{equation}

The distance $\len{X_1 M}$, interpreted as part of the triangle $\triangle X_1 M' M$ can be calculated using the pythagorean theorem:

\begin{equation} \len{X_1 M'}: \len{X_1 M'}^2 + m^2 = r^2 \end{equation}

The length $\len{AM}$ of the vector $\vec{AM}$ can be calculated using the scalar product, i.e. projecting the vector on itself, and taking the square root:

\begin{equation} \len{AM} = \sqrt{\vec{AM} \cdot \vec{AM}} \end{equation}

Using both pythagorean equations, eliminating $m$, one finds:

\begin{equation}\begin{split}
  \len{X_1 M'} &= \sqrt{r^2 - m^2} \\
    &= \sqrt{r^2 - (\len{AM}^2 - \len{AM'}^2)} \\
    &= \sqrt{r^2 - \len{AM}^2 + \len{AM'}^2} \\
    &= \sqrt{r^2 - \vec{AM} \cdot \vec{AM} + \len{AM'}^2}
\end{split}\end{equation}

\paragraph{Algorithm} Using these relations, the desired distances $\len{AX_1}, \len{AX_2}$ can be calculated with the following steps:

1. Calculate the length $\len{AM'}$ by projecting $\vec{AM}$ onto the unit vector $\vec{v}$ in $\vec{AB}$-direction: $$ \len{AM'} = \vec{AM} \cdot \vec{v} $$
1. Calculate $\len{X_1M'}^2$: $$ \len{X_1 M'}^2 = r^2 - \vec{AM} \cdot \vec{AM} + \len{AM'}^2 $$
1. Calculate $\len{X_1M'}$: $$ \len{X_1 M'} = \sqrt{\len{X_1 M'}^2} $$ In this approach, $\len{X_1M'}^2$ plays the role of the **discriminant**. If it is greater than zero, there are two intersection points. If it is zero, the intersection point $X$ falls onto $M'$, i.e. the line is actually a tangent. If the discriminant is less than zero, there is no intersection, because the radius $r$ is too small, i.e. the line is out of reach, resulting in the discriminant becoming negative.
1. Calculate the desired distances $\len{AX_1}$ and $\len{AX_2}$: $$ \len{A X_{1,2}} = \len{A M'} \mp \len{X_1 M'} $$

Or, as C code using the opencl native vector functions:

\begin{ccode}
  // Step 1
  const floating4_t vector_AM = (const floating4_t)(M.x - A.x, M.y - A.y, 0, 0);
  const floating_t length_AMprime = dot(vector_AM, v);

  // Step 2
  const floating_t discriminant = r * r - dot(vector_AM, vector_AM) + length_AMprime * length_AMprime;

  // Step 3
  const floating_t length_XMprime = native_sqrt(discriminant);

  // Step 4
  const floating_t length_AX1 = length_AMprime - length_XMprime;
  const floating_t length_AX2 = length_AMprime + length_XMprime;
\end{ccode}


