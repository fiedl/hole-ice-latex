%!TEX TS-program = ../make.zsh

\section{Theoretical Background}
\label{sec:theoretical_background}

## Neutrinos

- \cite{lexikonderphysik}

\smallerimage{standard-model} \cite{standardmodelwiki}

- mass eigenstates $\Pnu_i$ are related to weak eigenstates $\Pnu_\Plepton$ via the neutrino mixing matrix $U_{\Plepton i}$: $$ \ket{\Pnu_\Plepton} = \sum_i U_{\Plepton i}\,\ket{\Pnu_i} $$ $i \in {1,2,3}$, $\Plepton \in {\Pe, \Pmu, \Ptau}$ \cite{particledatareview}

## Neutrino Interactions Relevant to IceCube

- track-like events, charged current interaction of high energy muon neutrino with nucleus, producing hadronic shower at the vertex and outgoing muon that emits cherenkov light in a cone along its track \cite{instrumentation}
- electromagnetic or hadronic showers from interactions of all neutrino flavors, resulting in spherical light generation \cite{instrumentation}

- \enquote{Three event topologies are taken into account when considering neutrino interactions in IceCube} \cite{skysearch}
  - tracks; at TeV muons travel long distances, larger than several kilometers in the antarctic ice; light constantly emitted along the track \cite{skysearch,mmc}
  - shower or cascade like events, charged current interactions of electron or tau neutrinos as well as neutral current interactions of any neutrino type \cite{skysearch}
  - double bang, very high energy charged current tau tau anti neutrino interactions \cite{skysearch}

- primary neutrino interaction channel is deep-inelastic scattering with nuclei in the detector material \cite{energyreco}
- neutral and charged-current interactions, shower of hadrons created at the neutrino interaction vertex \cite{energyreco}
- in charged-current interactions this is accompanied by an outgoing charged lepton \cite{energyreco}
- cherenkov light is radiated by this primary lepton and in showers \cite{energyreco}

- double bang: two cascades joind by a short track \cite{energyreco}

- $\nu_e + N \rightarrow e + had$, charged current interaction, cascade signature \cite{energyreco}
- $\nu_\mu + N \rightarrow \mu + had$, charged current interaction, track signatur, optional cascade \cite{energyreco}
- $\nu_\tau + N \rightarrow \tau + had \rightarrow had$, charged current interaction, cascade, double bang signature \cite{energyreco}
- $\nu_\tau + N \rightarrow \tau + had \rightarrow \mu + had$, charged current interaction, cascade + track signature \cite{energyreco}
- $\nu_l + N \rightarrow \nu_l + had$, neutral current, cascade \cite{energyreco}

\smallerimage{evidence2013-event-20} 1141 TeV shower, cascade completely contained \cite{evidence2013,energyreco}
\smallerimage{evidence2013-event-5} mu started in the detector, deposited 71 TeV before leaving, track \cite{evidence2013,energyreco}

\smallerimage{feynman-neutral}
\smallerimage{feynman-charged}

## Cherenkov-light Emission
- wave length: visible, near uv
- see \cite{ppcpaper}, section 1

- charged particles resulting from neutrino interactions move through the ice faster than the phase velocity of light in ice, and therefore emit Cherenkov photons \cite{instrumentation}
- per GeV of secondary particle shower energy, order of $10^5$ visible Cherenkov photons are created \cite{instrumentation}
- produced by relativistic charged secondary particles \cite{skysearch}

\subsection{Photon Absorption and Scattering}
\label{sec:scattering}

The propagation of light through a medium depends on the optical properties of that medium, in particular the velocity of light within that medium, the scattering probability and the absorption probability. \cite{lundberg}
% lundberg, page 4

The absorption of light for the relevant wavelength range is caused by electronic and molecular excitation processes \cite{lundberg}
% lundberg, page 4
and is quantified by the \textbf{absorption length} $\lambda\abs$, which is the mean of the exponentially distributed free path length to absorption \cite{lundberg}.
% lundberg
Therefore, in accordance with the \noun{Beer-Lambert law}, the absorption length is the path length that light needs to travel within a medium to have its intensity drop to $\sfrac{1}{\e}$ of its original intensity. \cite{lexikonderphysik}
% lexikonderphysik, Band 1, Absorptionslänge
% Absorption lengths in the south-polar ice are in the order of $100\m$ and can in clear ice layers in the absence of dust even exceed $200\m$. \cite{ackermann}
% ackermann, page 5, paragraph 12
Absorption lengths in the south-polar ice vary between $20\m$ in dusty regions and $280\m$ in very clear ice layers. \cite{ackermann}\cite{ppcpaper}\cite{icepaper}

The scattering of light off microscopic scattering centers, such as submillimeter-sized air bubbles and micron-sized dust grains \cite{Price1997}\cite{ackermann} is the dominant scattering mechanism in glacial ice \cite{Askebjer1997}\cite{lundberg}. This scattering can be modelled using the more general \noun{Mie scattering} theory, which describes the scattering of electromagnetic radiation off small, spherical masses of material with refractive indices differing from the refractive index of its surroundings. \cite{Mie1908}\cite{ackermann}\cite{lundberg}

\noun{Mie scattering} gives the distribution of the scattering angle $\theta$ for any wavelength and scattering center size. For ice, this distribution is approximated using a one-parameter \noun{Henyey-Greenstein} (HG) phase function $p_\text{HG}(\theta; \tau)$, where the one parameter $\tau$ is the mean cosine of the scattering angle. \cite{lundberg}

\newcommand\meancostheta{\langle \cos \theta \rangle}
$$ p_\text{HG}(\theta; \tau) = \frac{1 - \tau^2}{2(1 + \tau^2 - 2\tau\, \cos \theta)^\frac{3}{2}}, \ \ \ \tau = \meancostheta $$

The south-polar ice has shown to be preferentially forward scattering with a mean cosine of the scattering angle of $\meancostheta = 0.94$ with only a weak dependence on the wavelength. \cite{ackermann}
% ackermann, paragraph 9

The \textbf{scattering length} $\lambda\sca$, which is also called \textbf{geometric} scattering length, is the mean of the exponentially distributed free path and thereby the average distance between scatterings. \cite{ackermann}

% https://github.com/fiedl/hole-ice-study/issues/52 - Effective scattering length
A related and often used quantity is the \textbf{effective scattering length} $\lambda\esca$, which is the distance that light needs to propagate through a turbid medium before the photon directions are completely randomized. \cite{lundberg}

\begin{equation}
  \lambda\esca = \frac{\lambda\sca}{1 - \meancostheta}
\end{equation}

In a medium with isotropic scattering, the geometric and the effective scattering length are the same. In a preferentially forward scattering medium like the south-polar ice, the original direction of a sample of photons is tendentially retained for several scattering steps until the photon direction of the sample is isotropized. The projection of the net velocity vector on the original direction is decreased on average by $\meancostheta$ for each scattering. \cite{lundberg}

After $n$ scatterings, the effectively transported forward distance along the original direction is $\lambda\sca\,\sum_{i=0}^n \meancostheta^i$, such that in the limit of many scatterings, $n \rightarrow \infty$, the effectively transported forward distance becomes the effective scattering length. \cite{lundberg}\cite{ackermann}

$$ \lim_{n \rightarrow \infty} \lambda\sca\,\sum_{i=0}^n \meancostheta^i = \frac{\lambda\sca}{1 - \meancostheta} = \lambda\esca $$

% Typical scattering length:
% average esca = 25m \cite{lundberg}
% scattering lengths between 5m and 90m in the detector volume \cite{ppcpaper}
% effective scattering coefficient for 400nm between 0.011/m and 0.2/m \cite{icepaper} fig. 16, corresponding to esca=5m to 90m esca

Typical effective scattering lengths within the south-polar ice are in the order of $25\m$ \cite{lundberg} and range from $5\m$ to $90\m$ in the detector volume \cite{icepaper}, corresponding to the geometric scattering length ranging from $30\cm$ to $5.4\m$.

% https://github.com/fiedl/hole-ice-study/issues/51 - point particles
Light interference effects are ignored during photon propagation as the average distance between the scattering centers is large compared to the photon wavelength. \cite{ackermann}

Also, despite modelling different ice regions with abrupt boundaries in the simulations of this study, the physical boundaries are assumed such that refractive index variations are continuous. Hence reflection at the medium boundaries is ignored in simulations. \cite{lundberg}
