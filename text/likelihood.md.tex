%!TEX root = ../diplomarbeit.tex

## Likelihood

When shooting photons onto a DOM, and counting how may photons are detected at this DOM, this can be interpreted as counting experiment where the probability $P(k)$ of counting $k$ photons at the DOM when emitting a total of $n$ photons, is given by the **binomial distribution** with the detection probability $p$.

$$ P_{p,n}(k) = \binom{n}{k}\,p^k\,(1-p)^{n-k} $$

When shooting photons from different angles $\theta_i$, and the detection probability under an angle $\theta$ is $p(\theta)$, the **likelihood** $L$ of detecting $k_i$ of the $n$ photons that have been started from angle $\theta_i$, is the joint probability of counting $k_i$ photons for each angle.

\begin{align*}
  L &= \prod_{i} P_{p(\theta_i),n}(k_i) \\
    &= \prod_{i} \binom{n}{k_i}\,p(\theta_i)^{k_i}\,(1-p(\theta_i))^{n-k_i}
\end{align*}

When studying the agreement of a reference curve $f(\theta)$ with the simulation data points $k_i$, the detection probablility $p(\theta)$ is given by the reference curve.

$$ p(\theta) = f(\theta) $$

When the same experiment is repeated under several sets of external conditions, parameter sets $\lambda_j$, each parameter set may lead to a different agreement $L_j$ of data points and reference curve.

When searching for the parameter set that corresponds to the best agreement of data points and reference curve, one is looking for the maximum likelihood.

$$ \frac{\partial L(\lambda)}{\partial \lambda} \stackrel{!}{=} 0 $$

When looking for the extremum, it's more convenient to use the logarithm of the likelihood rather than the likelihood itself, which preserves the monotonicity behaviour and the position of the extrema.

$$ \frac{\partial \ln L(\lambda)}{\partial \lambda} \stackrel{!}{=} 0 $$

When taking the logarithm of the above likelihood $L$, the product becomes a sum and the exponents become factors.

\begin{align*}
  \ln L &= \sum_i \left[ \ln \binom{n}{k} + \ln p(\theta_i)^{k_i} + \ln\,(1 - p(\theta_i))^{n-k_i} \right] \\
        &= \sum_i \left[ \ln \binom{n}{k} + k_i\,\ln p(\theta_i) + (n-k_i)\,\ln\,(1 - p(\theta_i)) \right]
\end{align*}


## Gauging

When the detection probability $p(\theta)$ is given by the reference curve $f(\theta)$ except for a constant factor $p_0$, then this factor $p_0$ needs to be determined by a gauging experiment.

$$ p(\theta) = p_0 \cdot f(\theta) $$

For example, if $p_0$ is the probability for a photon to reach the DOM regardless of the angle $\theta$ but depending on parameters like the shooting distance, then $p_0$ can be obtained from an experiment that is designed such that $f(\theta) = 1 \ \forall \theta$.

$$ p_0 = p(\theta) \underset{n \to \infty}{\rightarrow} \frac{k}{n} $$

