
## Hole Ice Trajectory Corrections

Consider a random photon trajectory through the ice. The photon is created by a particle interaction at the starting point of the trajectory. On its way, the photon is scattered several times within the ice before it is absorbed and thereby detected in one of IceCube's DOMs.

\image{photon-trajectory-oheeL3ai.pdf}

- Consider path of photon through ice.
- Interaction propability determined by ice properties and photon wavelength.
- Interaction points determined by these properties and random process.
- Look at part of this trajectory, A B between two interactions.
- Change scenario by adding hole ice cylinder.
- Determine how the trajectory is affected by this.


- Consider a photon trajectory through the ice that partially goes through hole ice.
- Within hole ice, the ice properties are different from outside.
- Most notably, the impurities (dirt) cause the photons to scatter or be absorbed more often there.
- I.e. Scattering and absorption coefficients are larger within hole ice.
- Correspondingly, the scattering and absorption lengths, which are the mean free distances until scattering or absorption, are shorter.

\image{photon-trajectory-aiph6ahD.pdf}

- Consider simulation of a photon.
- $A$ is the last point of interaction, i.e. fixed.
- $B$ is determined by random process.
- Consider a scenario where a photon trajectory starts at point $A$ and ends at point $B$, where it is scattered or absorbed. The photon does not interact inbetween $A$ and $B$.

\image{photon-trajectory-Edahi9sh.pdf}

- Now, change scenario: add hole ice cylinder.
- Due to locally increased interaction coefficients, the free path is shorter, i.e. modified $B$.
- Why is the path shorter?
- B is determined by random variable.
- If the mean free path is shorter (because the likelyhood if interaction is larger), then the random variable is smaller.
- The change is $\Delta b:= \overline{AB'} - \overline{AB}$.

