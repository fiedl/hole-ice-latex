%!TEX root = ../diplomarbeit.tex

\subsection{Source code of Algorithm B}
\label{sec:algorithm_b_source}

This is the source code of the second propagation algorithm through different media, described in section \ref{sec:algorithm_b}.

\sourcepar{The source code can also be found online at \url{https://github.com/fiedl/clsim/tree/sf/hole-ice-2018/resources/kernels/lib}.}

\begin{ccode}
// https://github.com/fiedl/clsim/blob/sf/hole-ice-2018/resources/kernels/lib/propagation_through_media/propagation_through_media.c

#ifndef PROPAGATION_THROUGH_MEDIA_C
#define PROPAGATION_THROUGH_MEDIA_C

#include "propagation_through_media.h"
#include "../ice_layers/ice_layers.c"
#ifdef HOLE_ICE
  #include "../hole_ice/hole_ice.c"
#endif


// PROPAGATION THROUGH DIFFERENT MEDIA 2018: Layers, Cylinders
// -----------------------------------------------------------------------------

// We know how many scattering lengths (`sca_step_left`) and
// absorption lengths (`abs_lens_left`) the photon will
// travel in this step.
//
// Because the mean scattering and absorption lengths are local
// properties, i.e. depend on the ice layer or whether the photon
// is within a hole-ice cylinder, we need to convert `sca_step_left`
// and `abs_lens_left` to geometrical distances in order to determine
// where the next interaction point is, i.e. how far to propagate
// the photon in this step.

inline void apply_propagation_through_different_media(
  floating4_t photonPosAndTime, floating4_t photonDirAndWlen,
  #ifdef HOLE_ICE
    unsigned int numberOfCylinders, __constant floating4_t *cylinderPositionsAndRadii,
    __constant floating_t *cylinderScatteringLengths, __constant floating_t *cylinderAbsorptionLengths,
  #endif
  floating_t *distances_to_medium_changes, floating_t *local_scattering_lengths, floating_t *local_absorption_lengths,
  floating_t *sca_step_left, floating_t *abs_lens_left,
  floating_t *distancePropagated, floating_t *distanceToAbsorption)
{

  int number_of_medium_changes = 0;
  distances_to_medium_changes[0] = 0.0;
  int currentPhotonLayer = min(max(findLayerForGivenZPos(photonPosAndTime.z), 0), MEDIUM_LAYERS-1);
  local_scattering_lengths[0] = getScatteringLength(currentPhotonLayer, photonDirAndWlen.w);
  local_absorption_lengths[0] = getAbsorptionLength(currentPhotonLayer, photonDirAndWlen.w);

  // To check which medium boundaries are in range, we need to estimate
  // how far the photon can travel in this step.
  //
  const floating_t photonRange = *sca_step_left * local_scattering_lengths[0];

  add_ice_layers_on_photon_path_to_medium_changes(
    photonPosAndTime,
    photonDirAndWlen,
    photonRange,

    // These values will be updates within this function:
    &number_of_medium_changes,
    distances_to_medium_changes,
    local_scattering_lengths,
    local_absorption_lengths
  );

  #ifdef HOLE_ICE
    add_hole_ice_cylinders_on_photon_path_to_medium_changes(
      photonPosAndTime,
      photonDirAndWlen,
      photonRange,
      numberOfCylinders,
      cylinderPositionsAndRadii,

      // These values will be updates within this function:
      &number_of_medium_changes,
      distances_to_medium_changes,
      local_scattering_lengths,
      local_absorption_lengths
    );
  #endif

  sort_medium_changes_by_ascending_distance(
    number_of_medium_changes,

    // These values will be updates within this function:
    distances_to_medium_changes,
    local_scattering_lengths,
    local_absorption_lengths
  );

  loop_over_media_and_calculate_geometrical_distances_up_to_the_next_scattering_point(
    number_of_medium_changes,
    distances_to_medium_changes,
    local_scattering_lengths,
    local_absorption_lengths,

    // These values will be updates within this function:
    sca_step_left,
    abs_lens_left,
    distancePropagated,
    distanceToAbsorption
  );
}

inline void sort_medium_changes_by_ascending_distance(int number_of_medium_changes, floating_t *distances_to_medium_changes, floating_t *local_scattering_lengths, floating_t *local_absorption_lengths)
{
  // Sort the arrays `distances_to_medium_changes`, `local_scattering_lengths` and
  // `local_absorption_lengths` by ascending distance to have the medium changes
  // in the right order.
  //
  // https://en.wikiversity.org/wiki/C_Source_Code/Sorting_array_in_ascending_and_descending_order
  //
  for (int k = 0; k <= number_of_medium_changes; k++) {
    for (int l = 0; l <= number_of_medium_changes; l++) {
      if (distances_to_medium_changes[l] > distances_to_medium_changes[k]) {
        floating_t tmp_distance = distances_to_medium_changes[k];
        floating_t tmp_scattering = local_scattering_lengths[k];
        floating_t tmp_absorption = local_absorption_lengths[k];

        distances_to_medium_changes[k] = distances_to_medium_changes[l];
        local_scattering_lengths[k] = local_scattering_lengths[l];
        local_absorption_lengths[k] = local_absorption_lengths[l];

        distances_to_medium_changes[l] = tmp_distance;
        local_scattering_lengths[l] = tmp_scattering;
        local_absorption_lengths[l] = tmp_absorption;
      }
    }
  }
}

inline void loop_over_media_and_calculate_geometrical_distances_up_to_the_next_scattering_point(int number_of_medium_changes, floating_t *distances_to_medium_changes, floating_t *local_scattering_lengths, floating_t *local_absorption_lengths, floating_t *sca_step_left, floating_t *abs_lens_left, floating_t *distancePropagated, floating_t *distanceToAbsorption)
{
  // We know how many scattering lengths (`sca_step_left`) and how many
  // absorption lengths (`abs_lens_left`) we may spend when propagating
  // through the different media.
  //
  // Convert these into the geometrical distances `distancePropagated` (scattering)
  // and `distanceToAbsorption` (absorption) and decrease `sca_step_left` and
  // `abs_lens_left` accordingly.
  //
  // Abort when the next scattering point is reached, i.e. `sca_step_left == 0`.
  // At this point, `abs_lens_left` may still be greater than zero, because
  // the photon may be scattered several times until it is absorbed.
  //
  for (int j = 0; (j < number_of_medium_changes) && (*sca_step_left > 0); j++) {
    floating_t max_distance_in_current_medium = distances_to_medium_changes[j+1] - distances_to_medium_changes[j];

    if (*sca_step_left * local_scattering_lengths[j] > max_distance_in_current_medium) {
      // The photon scatters after leaving this medium.
      *sca_step_left -= my_divide(max_distance_in_current_medium, local_scattering_lengths[j]);
      *distancePropagated += max_distance_in_current_medium;
    } else {
      // The photon scatters within this medium.
      max_distance_in_current_medium = *sca_step_left * local_scattering_lengths[j];
      *distancePropagated += max_distance_in_current_medium;
      *sca_step_left = 0;
    }

    if (*abs_lens_left * local_absorption_lengths[j] > max_distance_in_current_medium) {
      // The photon is absorbed after leaving this medium.
      *abs_lens_left -= my_divide(max_distance_in_current_medium, local_absorption_lengths[j]);
      *distanceToAbsorption += max_distance_in_current_medium;
    } else {
      // The photon is absorbed within this medium.
      *distanceToAbsorption += *abs_lens_left * local_absorption_lengths[j];
      *abs_lens_left = 0;
    }
  }

  // Spend the rest of the budget with the last medium properties.
  if (*sca_step_left > 0) {
    *distancePropagated += *sca_step_left * local_scattering_lengths[number_of_medium_changes];
    *distanceToAbsorption += *abs_lens_left * local_absorption_lengths[number_of_medium_changes];
    *abs_lens_left -= my_divide(*distancePropagated, local_absorption_lengths[number_of_medium_changes]);
  }

  // If the photon is absorbed, only propagate up to the absorption point.
  if (*distanceToAbsorption < *distancePropagated) {
    *distancePropagated = *distanceToAbsorption;
    *distanceToAbsorption = ZERO;
    *abs_lens_left = ZERO;
  }
}

#endif
\end{ccode}

\begin{ccode}
// https://github.com/fiedl/clsim/blob/sf/hole-ice-2018/resources/kernels/lib/propagation_through_media/propagation_through_media.h

#ifndef PROPAGATION_THROUGH_MEDIA_H
#define PROPAGATION_THROUGH_MEDIA_H

inline void apply_propagation_through_different_media(
  floating4_t photonPosAndTime, floating4_t photonDirAndWlen,
  #ifdef HOLE_ICE
    unsigned int numberOfCylinders, __constant floating4_t *cylinderPositionsAndRadii,
    __constant floating_t *cylinderScatteringLengths, __constant floating_t *cylinderAbsorptionLengths,
  #endif
  floating_t *distances_to_medium_changes, floating_t *local_scattering_lengths, floating_t *local_absorption_lengths,
  floating_t *sca_step_left, floating_t *abs_lens_left,
  floating_t *distancePropagated, floating_t *distanceToAbsorption);

inline void sort_medium_changes_by_ascending_distance(int number_of_medium_changes, floating_t *distances_to_medium_changes, floating_t *local_scattering_lengths, floating_t *local_absorption_lengths);

inline void loop_over_media_and_calculate_geometrical_distances_up_to_the_next_scattering_point(int number_of_medium_changes, floating_t *distances_to_medium_changes, floating_t *local_scattering_lengths, floating_t *local_absorption_lengths, floating_t *sca_step_left, floating_t *abs_lens_left, floating_t *distancePropagated, floating_t *distanceToAbsorption);

#endif
\end{ccode}

\begin{ccode}
// https://github.com/fiedl/clsim/blob/sf/hole-ice-2018/resources/kernels/lib/ice_layers/ice_layers.c

#ifndef ICE_LAYERS_C
#define ICE_LAYERS_C

#include "ice_layers.h"

inline void add_ice_layers_on_photon_path_to_medium_changes(floating4_t photonPosAndTime, floating4_t photonDirAndWlen, floating_t photonRange, int *number_of_medium_changes, floating_t *distances_to_medium_changes, floating_t *local_scattering_lengths, floating_t *local_absorption_lengths)
{

  // The closest ice layer is special, because we need to check how far
  // it is away from the photon. After that, all photon layers are equidistant.
  //
  floating_t z_of_closest_ice_layer_boundary =
      mediumLayerBoundary(photon_layer(photonPosAndTime.z));
  if (photonDirAndWlen.z > ZERO) z_of_closest_ice_layer_boundary +=
      (floating_t)MEDIUM_LAYER_THICKNESS;

  *number_of_medium_changes += 1;
  distances_to_medium_changes[*number_of_medium_changes] =
      my_divide(z_of_closest_ice_layer_boundary - photonPosAndTime.z, photonDirAndWlen.z);
  int next_photon_layer =
      photon_layer(z_of_closest_ice_layer_boundary + photonDirAndWlen.z);
  local_scattering_lengths[*number_of_medium_changes] =
      getScatteringLength(next_photon_layer, photonDirAndWlen.w);
  local_absorption_lengths[*number_of_medium_changes] =
      getAbsorptionLength(next_photon_layer, photonDirAndWlen.w);

  // Now loop through the equidistant layers in range.
  //
  const floating_t max_trajectory_length_between_two_layers =
      my_divide((floating_t)MEDIUM_LAYER_THICKNESS, my_fabs(photonDirAndWlen.z));
  while (distances_to_medium_changes[*number_of_medium_changes] + max_trajectory_length_between_two_layers < photonRange)
  {
    *number_of_medium_changes += 1;
    distances_to_medium_changes[*number_of_medium_changes] =
        distances_to_medium_changes[*number_of_medium_changes - 1]
        + max_trajectory_length_between_two_layers;
    next_photon_layer = photon_layer(photonPosAndTime.z
        + (distances_to_medium_changes[*number_of_medium_changes] + 0.01) * photonDirAndWlen.z);
    local_scattering_lengths[*number_of_medium_changes] =
        getScatteringLength(next_photon_layer, photonDirAndWlen.w);
    local_absorption_lengths[*number_of_medium_changes] =
        getAbsorptionLength(next_photon_layer, photonDirAndWlen.w);
  }

}

inline int photon_layer(floating_t z)
{
  return min(max(findLayerForGivenZPos(z), 0), MEDIUM_LAYERS-1);
}

#endif
\end{ccode}

\begin{ccode}
// https://github.com/fiedl/clsim/blob/sf/hole-ice-2018/resources/kernels/lib/ice_layers/ice_layers.h

#ifndef ICE_LAYERS_H
#define ICE_LAYERS_H

inline void add_ice_layers_on_photon_path_to_medium_changes(floating4_t photonPosAndTime, floating4_t photonDirAndWlen, floating_t photonRange, int *number_of_medium_changes, floating_t *distances_to_medium_changes, floating_t *local_scattering_lengths, floating_t *local_absorption_lengths);

inline int photon_layer(floating_t z);

#endif
\end{ccode}

\begin{ccode}
// https://github.com/fiedl/clsim/blob/sf/hole-ice-2018/resources/kernels/lib/hole_ice/hole_ice.c

#ifndef HOLE_ICE_C
#define HOLE_ICE_C

#include "hole_ice.h"
#include "../intersection/intersection.c"

inline void add_hole_ice_cylinders_on_photon_path_to_medium_changes(floating4_t photonPosAndTime, floating4_t photonDirAndWlen, floating_t photonRange, unsigned int numberOfCylinders, __constant floating4_t *cylinderPositionsAndRadii, int *number_of_medium_changes, floating_t *distances_to_medium_changes, floating_t *local_scattering_lengths, floating_t *local_absorption_lengths)
{
  // Find out which cylinders are in range in a separate loop
  // in order to improve parallelism and thereby performance.
  //
  // See: https://github.com/fiedl/hole-ice-study/issues/30
  //
  #ifdef NUMBER_OF_CYLINDERS
    // When running this on OpenCL, defining arrays using a constant
    // as array size is not possible. Therefore, we need to use a
    // pre-processor makro here.
    //
    // See: https://github.com/fiedl/hole-ice-study/issues/38
    //
    int indices_of_cylinders_in_range[NUMBER_OF_CYLINDERS];
  #else
    int indices_of_cylinders_in_range[numberOfCylinders];
  #endif
  {
    unsigned int j = 0;
    for (unsigned int i = 0; i < numberOfCylinders; i++) {
      indices_of_cylinders_in_range[i] = -1;
    }
    for (unsigned int i = 0; i < numberOfCylinders; i++) {
      if (sqr(photonPosAndTime.x - cylinderPositionsAndRadii[i].x) +
          sqr(photonPosAndTime.y - cylinderPositionsAndRadii[i].y) <=
          sqr(photonRange + cylinderPositionsAndRadii[i].w /* radius */))
      {

        // If the cylinder has a z-range check if we consider that cylinder
        // to be in range. https://github.com/fiedl/hole-ice-study/issues/34
        //
        if ((cylinderPositionsAndRadii[i].z == 0) || ((cylinderPositionsAndRadii[i].z != 0) && !(((photonPosAndTime.z < cylinderPositionsAndRadii[i].z - 0.5) && (photonPosAndTime.z + photonRange * photonDirAndWlen.z < cylinderPositionsAndRadii[i].z - 0.5)) || ((photonPosAndTime.z > cylinderPositionsAndRadii[i].z + 0.5) && (photonPosAndTime.z + photonRange * photonDirAndWlen.z > cylinderPositionsAndRadii[i].z + 0.5)))))
        {
          indices_of_cylinders_in_range[j] = i;
          j += 1;
        }
      }
    }
  }

  // Now loop over all cylinders in range and calculate corrections
  // for `*distancePropagated` and `*distanceToAbsorption`.
  //
  for (unsigned int j = 0; j < numberOfCylinders; j++) {
    const int i = indices_of_cylinders_in_range[j];
    if (i == -1) {
      break;
    } else {

      IntersectionProblemParameters_t p = {

        // Input values
        photonPosAndTime.x,
        photonPosAndTime.y,
        cylinderPositionsAndRadii[i].x,
        cylinderPositionsAndRadii[i].y,
        cylinderPositionsAndRadii[i].w, // radius
        photonDirAndWlen,
        1.0, // distance used to calculate s1 and s2 relative to

        // Output values (will be calculated)
        0, // discriminant
        0, // s1
        0  // s2

      };

      calculate_intersections(&p);

      if (intersection_discriminant(p) > 0) {
        if ((intersection_s1(p) <= 0) && (intersection_s2(p) >= 0)) {
          // The photon is already within the hole ice.
          local_scattering_lengths[0] = cylinderScatteringLengths[i];
          local_absorption_lengths[0] = cylinderAbsorptionLengths[i];
        } else if (intersection_s1(p) > 0) {
          // The photon enters the hole ice on its way.
          *number_of_medium_changes += 1;
          distances_to_medium_changes[*number_of_medium_changes] = intersection_s1(p);
          local_scattering_lengths[*number_of_medium_changes] = cylinderScatteringLengths[i];
          local_absorption_lengths[*number_of_medium_changes] = cylinderAbsorptionLengths[i];
        }
        if (intersection_s2(p) > 0) {
          // The photon leaves the hole ice on its way.
          *number_of_medium_changes += 1;
          distances_to_medium_changes[*number_of_medium_changes] = intersection_s2(p);
          if (i == 0) // there is no larger cylinder
          {
            const int photonLayerAtTheCylinderBorder =
                photon_layer(photonPosAndTime.z + photonDirAndWlen.z * intersection_s2(p));
            local_scattering_lengths[*number_of_medium_changes] =
                getScatteringLength(photonLayerAtTheCylinderBorder, photonDirAndWlen.w);
            local_absorption_lengths[*number_of_medium_changes] =
                getAbsorptionLength(photonLayerAtTheCylinderBorder, photonDirAndWlen.w);
          } else {
            // There is a larger cylinder outside this one, which is the one before in the array.
            // See: https://github.com/fiedl/hole-ice-study/issues/47
            //
            local_scattering_lengths[*number_of_medium_changes] = cylinderScatteringLengths[i - 1];
            local_absorption_lengths[*number_of_medium_changes] = cylinderAbsorptionLengths[i - 1];
          }
        }
      }

    }
  }

}

#endif
\end{ccode}

\begin{ccode}
// https://github.com/fiedl/clsim/blob/sf/hole-ice-2018/resources/kernels/lib/hole_ice/hole_ice.h

#ifndef HOLE_ICE_H
#define HOLE_ICE_H

inline void add_hole_ice_cylinders_on_photon_path_to_medium_changes(floating4_t photonPosAndTime, floating4_t photonDirAndWlen, floating_t photonRange, unsigned int numberOfCylinders, __constant floating4_t *cylinderPositionsAndRadii, int *number_of_medium_changes, floating_t *distances_to_medium_changes, floating_t *local_scattering_lengths, floating_t *local_absorption_lengths);

#endif
\end{ccode}

\begin{ccode}
// https://github.com/fiedl/clsim/blob/sf/hole-ice-2018/resources/kernels/lib/intersection/intersection.c

#include "intersection.h"

inline void calculate_intersections(IntersectionProblemParameters_t *p)
{
  // Step 1
  const floating4_t vector_AM = {p->mx - p->ax, p->my - p->ay, 0.0, 0.0};
  const floating_t xy_projection_factor = my_sqrt(1 - sqr(p->direction.z));
  const floating_t length_AMprime = dot(vector_AM, p->direction) / xy_projection_factor;

  // Step 2
  p->discriminant = sqr(p->r) - dot(vector_AM, vector_AM) + sqr(length_AMprime);

  // Step 3
  const floating_t length_XMprime = my_sqrt(p->discriminant);

  // Step 4
  const floating_t length_AX1 = length_AMprime - length_XMprime;
  const floating_t length_AX2 = length_AMprime + length_XMprime;
  p->s1 = length_AX1 / p->distance / xy_projection_factor;
  p->s2 = length_AX2 / p->distance / xy_projection_factor;
}

inline floating_t intersection_s1(IntersectionProblemParameters_t p)
{
  return p.s1;
}

inline floating_t intersection_s2(IntersectionProblemParameters_t p)
{
  return p.s2;
}

inline floating_t intersection_discriminant(IntersectionProblemParameters_t p)
{
  return p.discriminant;
}

inline floating_t intersection_x1(IntersectionProblemParameters_t p)
{
  if ((p.s1 > 0) && (p.s1 < 1))
    return p.ax + p.direction.x * p.distance * p.s1;
  else
    return my_nan();
}

inline floating_t intersection_x2(IntersectionProblemParameters_t p)
{
  if ((p.s2 > 0) && (p.s2 < 1))
    return p.ax + p.direction.x * p.distance * p.s2;
  else
    return my_nan();
}

inline floating_t intersection_y1(IntersectionProblemParameters_t p)
{
  if ((p.s1 > 0) && (p.s1 < 1))
    return p.ay + p.direction.y * p.distance * p.s1;
  else
    return my_nan();
}

inline floating_t intersection_y2(IntersectionProblemParameters_t p)
{
  if ((p.s2 > 0) && (p.s2 < 1))
    return p.ay + p.direction.y * p.distance * p.s2;
  else
    return my_nan();
}

inline bool intersecting_trajectory_starts_inside(IntersectionProblemParameters_t p)
{
  return (intersection_s1(p) <= 0) &&
      (intersection_s2(p) > 0) &&
      (intersection_discriminant(p) > 0);
}

inline bool intersecting_trajectory_starts_outside(IntersectionProblemParameters_t p)
{
  return ( ! intersecting_trajectory_starts_inside(p));
}

inline bool intersecting_trajectory_ends_inside(IntersectionProblemParameters_t p)
{
  return (intersection_s1(p) < 1) &&
      (intersection_s2(p) >= 1) &&
      (intersection_discriminant(p) > 0);
}
\end{ccode}

\begin{ccode}
// https://github.com/fiedl/clsim/blob/sf/hole-ice-2018/resources/kernels/lib/intersection/intersection.h

#ifndef INTERSECTION_H
#define INTERSECTION_H

typedef struct IntersectionProblemParameters {

  // Input values
  //
  floating_t ax;
  floating_t ay;
  floating_t mx;
  floating_t my;
  floating_t r;
  floating4_t direction;
  floating_t distance;

  // Output values, which will be calculated in
  // `calculate_intersections()`.
  //
  floating_t discriminant;
  floating_t s1;
  floating_t s2;

} IntersectionProblemParameters_t;

#endif
\end{ccode}