%!TEX TS-program = ../make.zsh

\subsection{Scanning the Angular Acceptance of an Optical Module}
\label{sec:angular_acceptance_scan}

IceCube's digital optical modules (DOMs) are more sensitive to accepting, that is to say to registering incoming photons from certain direction as from others. This is due to the detector design as the photomultiplier tube (PMT) is facing downwards as illustrated in figure \ref{fig:weihee6X} (a). Therefore, a photon coming from below is more likely to hit the PMT and be detected than a photon coming from above. Also the quantum efficiency of the PMT depends on the impact angle of the incoming photon.

\todo{quantum efficiency: this is more general: also glas and gel effects are included in the nominal curve. see 2016-02-18.}

Figure \ref{fig:weihee6X} (b) shows the measured \textbf{angular acceptance} $a\dom(\eta)$ of the IceCube DOMs by plotting the \textbf{relative sensitivity} from lab measurements for each polar angle representing the direction of incoming photons. \cite{icepaper} The \textbf{sensitivity} is the fraction of the incoming photons that are registered by the detector. The sensitivity is normalized \textbf{relative} to the optimal incoming angle, which is from below, where the relative sensitivity is defined to be 1. Note that the polar angle $\eta$ is measured from below: For photons coming from below, $\eta = 0$. For photons coming from above, $\eta = \pi$.

\question{Are sensitivity and acceptance the same?}

\begin{figure}[htbp]
  \subcaptionbox{Side view of a IceCube digital optical module (DOM). The photomultiplier tube (PMT) in each DOM is facing downwards. Incoming photons are shown in the left upper cornder of the illustration. They are moving towards the DOM under a polar angle $\eta$ measured from below. Image taken from \cite{martindardupdate}, slide 9.}{\halfimage{angular-acceptance-schematics-martin-feeJ0yah}}\hfill
  \subcaptionbox{Angular acceptance $a(\eta)$: Sensitivity of the DOM for registering incoming photons depending on the polar angle $\eta$ relative to the sensitivity for registering photons from the optimal incoming angle $\eta = 0, \cos \eta = 1$. The ``nominal'' curve $a\dom(\eta)$ does not consider hole-ice effects and is based on lab measurements. The ``hole ice'' curve $a\domhi(\eta)$ is based on previous simulations using the \noun{Photonics} software \cite{lundberg} approximating the effects of the hole ice on the sensitivity for registering incoming photons. Plot taken from \cite{icepaper}, figure 7.}{\halfimage{ice-paper-fig7}}
  \caption{Angular acceptance: The sensitivity of IceCube's digital optical modules (DOMs) depends on the polar angle $\eta$ of the direction of incoming photons relative to the DOM.}
  \label{fig:weihee6X}
\end{figure}

\label{sec:hole_ice_effects}
When considering the \textbf{effect of hole ice on the detection of incoming photons}, the effect is considered to depend on the polar angle $\eta$ of the incoming photons. When the photons are coming from the side, $\eta = \sfrac{\pi}{2}, \cos \eta = 0$, the distance the incoming photons need to travel through the hole ice is minimal. Therefore, the effect of the hole ice is expected to be minimal from this direction.

For photons approaching the optical module from below, $\eta = 0, \cos \eta = 1$, the hole-ice effect should be strong as the distance the photons need to travel through the hole ice is maximal. The photons are likely to be scattered away before reaching the optical module, effectively reducing the sensitivity for photons approaching the optical module from below as shown in figure \ref{fig:weihee6X} (b) by the blue curve in the region of $\cos \eta = 1$.

For photons approaching the optical module from above, $\eta = \pi, \cos \eta = -1$, the hole-ice effect should also be strong as the distance the photons need to travel throught he hole ice is maximal. The optical module is very insensitive to registering photons coming from above. But with the hole ice, photons approaching the optical module from above are likely to be scattered away before reaching the optical module. There is a chance that those photons travel around the optical module and finally hit the detector at a lower position in the sensitive area. This effectively increases the sensitivity for photons approaching the optical module from above as shown in figure \ref{fig:weihee6X} (b) by the blue curve in the region of $\cos \eta = -1$.


\paragraph{Measuring the Angular Acceptance With Simulations}
In order to quantify the hole-ice effect on the angular acceptance of IceCube's optical modules for different hole ice models, this study performs a series of simulations, starting photons from different angles $\eta_i$ towards an optical module and counting the photons registered by the optical module.

The simulation records the number $N(\eta_i):= N$ of started photons for each angle $\eta_i$, as well as the number $k(\eta_i)$ of registered hits for each angle. The relative hit frequency is $h(\eta_i)$.

\begin{equation}
  h(\eta_i) = \frac{k(\eta_i)}{N}
\end{equation}

To make this relative frequency $h(\eta_i)$ comparable to the DOM's angular acceptance $a\dom(\eta)$, which is defined such that $a\dom(\eta = 0) = 1$, this study often uses a renormalized relative frequency $\tilde{h}(\eta_i):=g\,h(\eta_i)$ with a gauging factor $g\in\reals$ such that $\tilde{h}(\eta = 0) = 1$.

\begin{equation}
  \tilde{h}(\eta_i) = g\,h(\eta_i), \ \ \
  g = \frac{1}{h(\eta = 0)}
  \label{eq:gauging_factor}
\end{equation}

\sourcepar{A script to configure and perform these kinds of simulations is provided in \script{AngularAcceptance}.}

\paragraph{Photon Sources: Pencil Beams and Plane Waves}
This study considers two types of photon sources for angular-acceptance studies, pencil beams and plane waves. Both are illustrated in figure \ref{fig:quie8Oof}.

In the case of pencil beams, one simulation is performed for each angle $\eta_i$. Photons are started from a distance $d$ from the center of the optical module at a fixed position such that it approaches the optical module under a polar angle $\eta_i$ as shown in figure \ref{fig:quie8Oof} (a).

In order to approximate plane waves as photon sources, photons are not started from fixed positions but from random positions within a plane that is located in a distance $d$ from the center of the optical module and oriented perpendicular to the distance vector as shown in figure \ref{fig:quie8Oof} (b).

Due to the limitation of computational resources, rather than simulating plane waves with infinite extend, an arbitrary extent $e$ of the plane is chosen. \authorname{Rongen} \cite{martindardupdate} compares the influence of the plane extent $e$ on the angular-acceptance curves (\cite{martindardupdate}, slides 6 and 9). The angular-acceptance simulations described in the section arbitrarily chosse a plane extent $e = 1\m$ and a start distance $d = 1\m$.

Note that the gauging factor $g$ (equation \ref{eq:gauging_factor}) depends on the photon source and needs to be redetermined when chaning the photon source, that is to say when switching between pencil beams, plane waves, or when changing the distance $d$, or the plane extent $e$.

Figure \ref{fig:Paihah7h} shows a \noun{Steamshovel} visualization of simulated photons started from a pencil beam, figure \ref{fig:Aehi7kae} for photons started from a plane.


% https://github.com/fiedl/hole-ice-study/issues/100
\begin{figure}[htbp]
  \subcaptionbox{Pencil beams: Start photons at fixed positions.}{\halfimage{angular-acceptance-coordinates-quie8Oof}}\hfill
  \subcaptionbox{Plane waves: Start photons from random positions within planes with extent $e$ approximating plane waves approaching the DOM. The two-dimensional planes are represented by lines in this $x$-$z$-projected view.}{\halfimage{angular-acceptance-coordinates-plane-waves-Ii2nieki}}
  \caption{Angular-acceptance scan: In each simulation, start photons from a different polar angle $\eta_i$ from a distance $d$ towards the digital optical module (DOM) and record the number of photons registered by the DOM. In this illustration, the DOM is viewed from the side.}
  \label{fig:quie8Oof}
\end{figure}

% https://github.com/fiedl/hole-ice-study/issues/98#issuecomment-409178654
\begin{figure}[htbp]
  \subcaptionbox{$2\ns$}{\halfimage{pencil-beam-2ns}}\hfill
  \subcaptionbox{$4\ns$}{\halfimage{pencil-beam-4ns}}\hfill
  \subcaptionbox{$9\ns$}{\halfimage{pencil-beam-9ns}}\hfill
  \subcaptionbox{$100\ns$}{\halfimage{pencil-beam-100ns}}
  \caption{\noun{Steamshovel} visualization of photons started as pencil beam under an angle of $\eta = \ang{45}$ from a distance of $d = 1\m$ towards the upmost optical module of a detector string. Snapshots are taken at $2\ns$, $4\ns$, $9\ns$, and $100\ns$ after starting the photons. The opening angle of the beam is $\ang{0.001}$. The photon spread seen in the event display is due to scattering along the photon trajectory and makes it necessary to gauge the simulation when chaning the starting distance $d$.}
  \label{fig:Paihah7h}
\end{figure}

% https://github.com/fiedl/hole-ice-study/issues/98#issuecomment-409120923
\begin{figure}[htbp]
  \subcaptionbox{$2\ns$}{\halfimage{plane-wave-2ns}}\hfill
  \subcaptionbox{$4\ns$}{\halfimage{plane-wave-4ns}}\hfill
  \subcaptionbox{$9\ns$}{\halfimage{plane-wave-9ns}}\hfill
  \subcaptionbox{$100\ns$}{\halfimage{plane-wave-100ns}}
  \caption{\noun{Steamshovel} visualization of photons started as plane wave with an extent of $e = 1\m$ under an angle of $\eta = \ang{45}$ from a distance of $d = 1\m$ towards the detector module. Snapshots are taken at $2\ns$, $4\ns$, $9\ns$, and $100\ns$ after starting the photons.}
  \label{fig:Aehi7kae}
\end{figure}


\paragraph{Acception Criterion: A Priori Angular Acceptance or Direct Detection}
In each simulation step of the photon-propagation simulation, the algorithm checks whether the photon intersects the sphere representing a optical module between two scattering points (see section \ref{sec:standard_photon_propagation_algorithm}). At this time, the intersection position and the direction of the photon are known. Based on that information, the simulation needs to decide whether the module accepts the hit, that is to say considers the incoming photon to be registered, or to ignore the hit in order to model the sensitivity of the optical module.

This study considers two approaches to this decision-making process: Accept the hits randomly based only on the measured \textbf{angular acceptance} $a\dom(\eta)$ of the optical module and the directions of the incoming photons, or accept the hits based only on the location of the sensitive photomultiplier area and the positions of the hits. The latter method is called \textbf{direct detection} \cite{martinspicehddard}. Both approaches are illustrated in figure \ref{fig:kieQuoh1}.

\begin{figure}[htbp]
  \centering
  \halfimage{direct-detection}
  \caption{Acception criteria: The DOM on the left-hand side accepts or rejects incoming photons as hits based only on the direction of the photon (``angular acception''). The DOM on the right-hand side accepts or rejects incoming photons only based on the impact position that determines whether the sensitive area of the photomultiplier tube has been hit (``direct detection''). Image taken from \cite{martinspicehddard}, slide 17.}
  \label{fig:kieQuoh1}
\end{figure}

Using only the photon direction to model the optical module's sensitivity, is the default approach in \noun{clsim}. This method has the advantage that both, the location of the photomultiplier tube, and its angular-dependent quantum efficiency, can be abstracted into a single angular-acceptance function that only depends on one parameter, the photon direction.

In \noun{clsim}, the angular acceptance $a\dom(\eta)$ of IceCube's optical modules has been implemented as polynomial.

\begin{equation}
  a\dom(\eta) = \sum_{j = 0}^{10} b_j\,\cos(\eta)^j, \ \ \ \eta \in [0; \pi]
\end{equation}

$$ b_0 = 0.26266, \ \ b_1 = 0.47659, \ \ b_2 = 0.15480, $$
$$ b_3 = -0.14588, \ \ b_4 = 0.17316, \ \ b_5 = 1.3070, $$
$$ b_6 = 0.44441, \ \ b_7 = -2.3538, \ \ b_8 = -1.3564, $$
$$ b_9 = 1.2098, \ \ b_{10} = 0.81569 $$

In this study, however, that focuses on the propagation through hole ice, the photons are expected to frequently scatter, that is to say change their direction in close proximity to the optical module. In this scenario, using the position of a hit rather than only the direction, is also of interest, in particular as, according to \authorname{Rongen} \cite{martinspicehddard}, direct detection is needed to distinguish different hole-ice models, and has been implemented for this study in \noun{clsim} as alternative to the standard angular-acception method.

\docpar{The implementation of direct detection is documented in \issue{32}.}

The most accurate approach to modeling the sensitivity of the optical modules to registering incoming photons would be to take both, the hit position, and the photon direction into account. The position would account for whether the photon would hit the sensitive area of the photomultiplier tube. The impact angle would account for the quantum efficiency of the tube, which depends on the impact angle. Combining both approaches in \noun{clsim}, however, is considered out of scope of this study.

Also, this study does not consider inclined orientations of the optical modules, but always assumes that the modules are oriented along the $z$-axis.\footnote{To check if DOM orientations have been implemented at the time of reading, check \url{https://github.com/fiedl/hole-ice-study/issues/53}.}

% Direct detection switch: https://github.com/fiedl/hole-ice-study/issues/32
% Consider DOM orientation: https://github.com/fiedl/hole-ice-study/issues/53


\paragraph{Angular-Acceptance Simulations Without Hole Ice}
In order to compare the different approaches that need to be considered when implementing simulations to scan the angular acceptance, that is to say whether to use pencil beams or plane waves as photon sources, and whether to use direct detection or the optical module's a priori angular acceptance $a\dom(\eta)$ to model the sensitivity, angular-acceptance simulations have been conducted using the new propagation algorithm, but without any hole-ice cylinders.

\docpar{These angular-acceptance simulations without hole ice have been documented in \issue{98}.}

Figure \ref{fig:Shai8yah} shows the results of the angular-acceptance scans, comparing the above approaches, each in comparison to the angular-acceptance curve $a\dom(\eta)$ from \cite{icepaper} (``nominal'' curve in figure \ref{fig:weihee6X}).

% https://github.com/fiedl/hole-ice-study/issues/98
\begin{figure}[htbp]
  \subcaptionbox{pencil beams, a priori angular acception}{\halfimage{angular-acceptance-pencil-beam-no-hole-ice-no-direct-detection}}\hfill
  \subcaptionbox{plane waves, a priori angular acception}{\halfimage{angular-acception-plane-waves-no-hole-ice-no-direct-detection}}
  \subcaptionbox{pencil beams, direct detection}{\halfimage{angular-acceptance-direct-detection-pencil-beam-no-hole-ice}}\hfill
  \subcaptionbox{plane waves, direct detection}{\halfimage{angular-acceptance-direct-detection-plane-waves-no-hole-ice}}\hfill
  \caption{Comparison of angular-acceptance-scan simulations with different approaches, all without hole ice. In (a), the simulation curve follows the a priori curve by design as the a priori curve is used as DOM acception criterion. With direct detection, (c) and (d), the a priori curve is not put into the simulation.}
  \label{fig:Shai8yah}
\end{figure}

The first simulation, using pencil beams and the a priori angular acceptance as acceptance criterion, figure \ref{fig:Shai8yah} (a), follows the a priori angular acceptance curve $a\dom(\eta)$ by design. When deactivating scattering and absorption in the bulk ice entirely, all photons would hit the optical module, and all photons would have still the original direction. All non-angular acceptance criteria would be absorbed in the gauging factor $g$ and both curves, the simulation curve and the a priori curve, would match exactly in the limit of many started photons $N\rightarrow\infty$.

When switching from pencil beams to plane waves, \ref{fig:Shai8yah} (b), more photons approaching the optical module from above, in the region of $\cos \eta = -1$, are accepted because there are photons that start from above, but shifted in the $x$-$y$ plane such that they will fly by the DOM and then be scattered and hit the DOM with an angle that is accepted by the DOM. This effect increases when increasing the extent $e$ of the plane waves.

\todo{Double check this by starting a plane wave with a smaller extent, for example 50cm.}

When switching to direct detection as acception criterion, figure \ref{fig:Shai8yah} (d), the a priori angular acceptance $a\dom(\eta)$ is no longer put into the simulation. Nevertheless, the simulation curve and the a priori curve have roughly the same shape. \authorname{Rongen} argues that the dominant factor for the shape of the angular-acceptance curve is determined by geometrical considerations \cite{martindardupdate}, which are reproduced by the simulation with direct detection.

Further studies could investigate whether finding the proper plane extent $e$ and starting distance $d$ would suffice to make both curves match.\followup

When switching to pencil beams with direct detection, figure \ref{fig:Shai8yah} (c), the simulation essentially becomes a scan for where the sensitive area of the photomultiplier tube ends in the optical module (compare figure \ref{fig:kieQuoh1}).


\paragraph{Angular-Acceptance Simulations With Hole Ice}
With the new propagation algorithm, the same simulations can be performed, gathering the relative frequencies $\tilde{h}(\eta_i)$ of photons started under an angle $\eta_i$ being accepted as hits by the optical module, but this time adding a hole-ice cylinder.

In a first attempt, arbitrary properties for the hole-ice cylinder are used, assuming a cylinder radius $r:=r\dom$ being the same as the radius $r\dom$ of the optical module, and assuming an arbitrary effective scattering length $\lambda\hi\esca = 1\m$.

\docpar{These angular-acceptance simulations with hole ice are documented in \issue{99}.}

Figure \ref{fig:eVapie9t} shows the results of the angular-acceptance scans with hole ice, comparing the different approaches (direct detection vs. angular acceptance, and plane waves vs. pencil beams), each in comparison to the a priori angular-acceptance curve $a\domhi(\eta)$ approximating the influence of hole ice from \cite{icepaper} (``hole ice'' curve in figure \ref{fig:weihee6X}).

\begin{equation}
  a\domhi(\eta) = \sum_{j = 0}^{10} b_j\,\cos(\eta)^j, \ \ \ \eta \in [0; \pi]
\end{equation}

$$ b_0 = 0.32813, \ \ b_1 = 0.63899, \ \ b_2 = 0.20049, $$
$$ b_3 = -1.2250, \ \ b_4 = -0.14470, \ \ b_5 = 4.1695, $$
$$ b_6 = 0.76898, \ \ b_7 = -5.8690, \ \ b_8 = -2.0939, $$
$$ b_9 = 2.3834, \ \ b_{10} = 1.0435 $$

% https://github.com/fiedl/hole-ice-study/issues/99
\begin{figure}[htbp]
  \subcaptionbox{pencil beams, a priori angular acception}{\halfimage{angular-acceptance-no-direct-detection-hole-ice-pencil-beam}}\hfill
  \subcaptionbox{plane waves, a priori angular acception}{\halfimage{angular-acceptance-no-direct-detection-hole-ice-plane-waves}}\hfill
  \subcaptionbox{pencil beams, direct detection}{\halfimage{angular-acceptance-direct-detection-hole-ice-pencil-beam}}\hfill
  \subcaptionbox{plane waves, direct detection}{\halfimage{angular-acceptance-direct-detection-hole-ice-plane-waves}}\hfill
  \caption{Comparison of angular-acception-scan simulations with different approaches, all with a hole-ice cylinder with arbitrary properties, assuming an effective hole-ice scattering length of $\lambda\hi_\e=1\m$ and a hole-ice-cylinder radius of $r=r\dom$.}
  \label{fig:eVapie9t}
\end{figure}

When using plane waves as photon sources, figure \ref{fig:eVapie9t} (b) and (d), the behaviour of photons approaching the optical module from below matches the results $a\domhi(\eta)$ from the previous simulations \cite{icepaper} already reasonably well for these hole-ice parameters.

The following simulations will use plane waves as photon sources and direct detection as acception criterion.

\todo{explane this choice. pencil beam does not look good with direct detection. direct detection feels like the better choice because no a priori angular acceptance is used, and the photons are expected to change their direction in close proximity to the DOM when scattering there within the hole ice.}
\question{Is this a good explanation?}


\paragraph{Varying the Hole-Ice-Cylinder Radius in Simulations}
With the new medium-propagation algorithm (section \ref{sec:algorithm_b}), the hole-ice parameters can be variied.

\docpar{Implementing angular-acceptance simulations for different hole-ice-cylinder radii is documented in \issue{82}.}

Figure \ref{fig:neiyi3Al} (a) shows angular-acceptance curves from simulations with different hole-ice-cylinder radii. In these simulations, the hole-ice cylinder's center is set to be the center of the optical module. The effective scattering length of the hole ice is fixed, $\lambda\esca\hi = 50\cm$.

\begin{figure}[htbp]
  \subcaptionbox{Varying the radius of the hole-ice cylinder.}{\halfimage{angular-acceptance-vary-radius-82}}\hfill
  \subcaptionbox{Varying the scattering length of the hole ice.}{\halfimage{angular-acceptance-vary-esca-83}}\hfill
  \caption{Comparison of angular-acceptance simulations with different hole-ice parameters. For larger hole-ice radii and smaller hole-ice scattering lengths, the angular-acceptance curves show a stronger influence of the hole ice on the detection of photons at the optical module. The blue curve shows the a priori angular-acceptance curve $a\domhi(\eta)$ from \cite{icepaper} that approximates the effect of the hole ice. LLH gives the log-likelihood value of comparing the simulation curve to the a priori curve using a binomial likelihood function.}
  \label{fig:neiyi3Al}
\end{figure}

When the hole-ice-cylinder radius is larger in the simulations, the hole-ice influence is larger because the ice volume occupied by the hole-ice cylinder is larger, making it more likely that photons approaching the detector module scatter within the hole ice.

When increasing the hole-ice-cylinder radius, photons approaching the optical module from above (left-hand side of figure \ref{fig:neiyi3Al} a) are increasingly scattered ``around'' the optical module such that they hit the optical module in the sensitive area at the bottom, resulting in more accepted hits.

Photons coming from below (right-hand side of figure \ref{fig:neiyi3Al} a) are increasingly scattered away before reaching the optical module when increasing the hole-ice-cylinder radius, resulting in less hits.


\paragraph{Varying the Hole-Ice-Scattering-Length in Simulations}
In another series of simulations, the scattering length of the propagating photons within the hole-ice cylinder is variied.

\docpar{Implementing angular-acceptance simulations for different hole-ice scattering lengths is documented in \issue{83}.}

Figure \ref{fig:neiyi3Al} (b) shows angular-acceptance curves from simulations with different hole-ice scattering lengths. In these simulations, the hole-ice cylinder's center is set to be the center of the optical module. The radius of the hole-ice cylinder is fixed to $r = 30\cm$.

When the scattering length within the hole-ice cylinder is shorter, photons propagating through the cylinder scatter more often.

For smaller scattering lengths, photons approaching the optical module from below (right-hand side of figure \ref{fig:neiyi3Al} b) are more likely to be scattered away in the hole ice before reaching the optical module, resulting in less hits.

On the left-hand side of figure \ref{fig:neiyi3Al} (b), where the photons are approaching the optical module from above, the effect of varying the scattering length is less prominent as compared to varying the cylinder radius (\ref{fig:neiyi3Al} a). The detection of photons approaching the optical module from above requires that photons flying by the optical module are scattered into the sensitive area of the module. For a plane wave of photons approaching the module from above, this is much more likely when increasing the hole-ice radius as compared to shortening the scattering length.

Both series of angular-acceptance simulations, varying the hole-ice scattering length and varying the hole-ice-cylinder radius, show the expected hole-ice effects (page \ref{sec:hole_ice_effects}).

\todo{Martin suggests in POCAM holice munich, slide 3, that the scattering length determines the position of the maximum, the size of the bubble column reduces the sensitivity in the forward region, right hand side of the plot}

As a next step, a series of angular-acceptance simulations are performed for hole-ice parameters suggested by other studies.

\paragraph{Simulation of Hole-Ice Parameters From ...}

- ice paper esca = 50cm, r = 30cm, #80

- albrecht karle's h2 model \cite{holeicestudieswithyag}: geometric scattering length L_scatt = 50cm, scattering length = geometric scattering length on air bubbles, hole ice radius = 30cm, based on measurements with a YAG-laser.

> These were derived many years ago, and are characterised by the scattering length of the bubble column. The H2 model is typically referred to as the "baseline" model, which corresponds to a scattering length of 50cm. In simulation, the effects of the H* models are parametrised via angular acceptance curves (see above) that determine the probability to accept a photon based on its arrival direction (not position) when it intersects with the DOM surface. The derivation of these angular acceptance curves assumed that the entire drill hole is full of the scattering centres. However from the Sweden camera images we suspect that this hypothesis is incorrect, and the bubbles instead concentrate in the centre of the drill hole rather than throughout.

- \todo{other hole-ice models: \url{https://wiki.icecube.wisc.edu/index.php/Hole_ice}}
- on slack, martin has argued why noone could kill the h2 for very long

> mrongen [7:53 PM 2018-03-26]
> Ok getting back to the bigger picture:
> The overall state of observations IMHO only leaves us with three plausible true realizations:
>  1. There is no bubble column
>  2. It is (usually) small and centered around the DOM
>  3. It is (usually) at the cable position
>
> These hypothesis (no bubble column being the null hypothesis against which the other two are tested) are actually fairly easy to test given the usual machinery. Use the new PPC with photon deletion on the cable, use direct detection (is that still in there?) and scan scattering length, size for both geometrical hypothesis (3 might need a small adjustment in PPC...) if the LLH improvement is significant that's it.
>
> A cross check for hypothesis 1 is also to see which scattering length (in a simulation without the cable and the column filling the entire drill hole) fits simulation data with the cable only. If it is 50cm that kind of explains why H2 has been so hard to kill (edited)

- Dima's Model, Angular acceptance through flasher unfolding, 2015
  \url{https://wiki.icecube.wisc.edu/index.php/MSU_Forward_Hole_Ice}
  \url{https://github.com/fiedl/hole-ice-study/issues/80#issuecomment-410478799}

- MSU Forward Hole Ice
  trying to combine H2 and Dima's model


- SpiceHD:

> See above. In principle, this is the best physically-motivated model, as photons are only accepted if they intersect the DOM in the active photocathode region, whereas angular acceptance models have the possibility to accept photons that intersect the top side of the DOM. Photons are directly propagated through a simulated bubble column. The parameters of the model are the size of the bubble column and the scattering length. However, it was discovered that this model actually found the location of the cable, not the bubble column. Once the cable position is finally determined from single LED data, this model can be revisited. But currently it is not advisable to use this model for simulation production used in high profile physics analyses. It should be considered only a test case/alternative model. It is not yet a realistic picture of the reality in the detector.

- POCAM munich paper, S.9, martin tests his direct detection
  martin simulates entire drill hole filled (r = 30cm) with esca=125cm, sca = 7.5cm
  which maches the ppc spline model and the h2 curve

- POCAM, S. 10, r = 0.6r_dom, esca = 14cm, between Dima and H2




\paragraph{Simulations for Hole-Ice Parameters From Other Studies}
- results from martin
- ice paper 2013: #80 -> check this issue for updates on why the do not match the reference curve
- further investigations are required to find out why the ice paper parameters do not match the ice paper a priori curve in this simuation setup
- first guess would be some systematic error in simulations, but cross checks look good
- maybe some systematic error in angular-acceptance-simulation setup

- also scan over parameters and compare agreement of a priori curve and simulation curve -> next section