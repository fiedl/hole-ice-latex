\section{Introduction}
\label{sec:intro}

\icecube is a neutrino observatory located at Earth's South Pole. It
uses a cubic-kilometer of glacial ice as detector medium where secondary
particles from neutrino interactions produce light as they move through
the ice. The light is detected by an array of photo detector modules
that are deployed throughout the ice. \cite{evidence2013}

The primary scientific objective of \icecube is the study of neutrinos
with energies ranging from \(10\TeV\) to \(10\PeV\) produced in
astrophysical processes, and the identification and characterization of
their sources. In collaboration with other neutrino detectors as
\noun{Antares}, with optical, x-ray, gamma-ray, radio, and
gravitational-wave observatories, \icecube participates in efforts for
multi-messenger astronomy. Other objectives include the indirect
detection of dark matter, the search for other exotic particles, and the
study of neutrino-oscillation physics.
\cite{instrumentation, evidence2013}

As \icecube detects neutrinos indirectly through the interaction with
other particles, involving a chain of processes and components, a key
requirement for precise measurements is to minimize uncertainties for
each process and component involved. Some components such as technical
instruments in the detector modules can be tested and calibrated in
isolation in laboratories. Other components involved such as the glacial
ice cannot be extracted and need to be studied where they are.
Uncertainties concerning the properties of the glacial ice can affect
the precision for measurements of the direction of the detected
neutrinos by several percent. \cite{wrede}

All light that is detected by the detector modules needs to travel
through the refrozen water of the drill holes that were needed to deploy
the detector modules within the ice. This so-called \textit{hole ice}
may have properties significantly different from the surrounding bulk
ice regarding the propagation of light through this medium. The
properties of the hole ice are less known than the properties of the
bulk ice and pose the largest systematic uncertainty for study of
neutrino oscillations and a number of other analyses. \cite{icrc17pocam}

This study aims to provide the necessary tools to improve detector
calibration by introducing the means to simulate the propagation of
light through the hole ice. By comparing different simulation scenarios,
involving hole ice of different respective properties, to calibration
data, it is then possible to study the properties of the hole ice and
its effect on the propagation of light, and on the detection of light by
the detector modules, reducing the systematic uncertainties imposed by
the hole ice, and in the long run improving the precision of the
\icecube observatory.

After providing some background information in sections
\ref{sec:theoretical_background} to \ref{sec:simulation_background}, two
algorithms and their integration into the existing \icecube software
framework will be presented in section \ref{sec:methods} that allow to
simulate the direct propagation of photons through hole ice of different
properties. The validity of the algorithms will be supported by a series
of tests and cross checks in section
\ref{sec:unit_tests_and_cross_checks}.

Examples of application such as the simulation of one or several
hole-ice cylinders with different sizes and photon scattering lengths,
the simulation of shadowing cables, and a calibration method using LED
flasher data are given in section\nbsp\ref{sec:applications}.
Section\nbsp\ref{sec:discussion} presents a brief comparison of methods
and preliminary results to other studies. This section also discusses
performance considerations and lists ice properties not considered in
this study.

The material needed to reproduce this study is provided on the
accompanying CD-ROM and can be found online at
\url{https://github.com/fiedl/hole-ice-study}.
