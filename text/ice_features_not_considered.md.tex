%!TEX TS-program = ../make.zsh

\subsection{Features Not Considered in This Study}
\label{sec:ice_features_not_considered}

This study provides the means for a \textit{direct propagation of photons through hole ice} and, in general, through cylinders of different ice properties (section \ref{sec:algorithm_b}). \textit{Direct detection} is added as acceptance criterion of the optical modules (section \ref{sec:direct_detection}).

For the propagation over large distances, the \textit{ice-layer tilt} and the \textit{ice anisotropy} (section \ref{sec:ice}) are omitted in this study and need to be re-implemented into the new medium-propagation algorithm.\footnote{For the current status of the re-implementation of ice-layer tilt and ice anisotropy, check \url{https://github.com/fiedl/hole-ice-study/issues/48}.}

The hole ice is modeled as homogeneous cylinder with exact boundaries. A gradient of the hole-ice properties in the radial direction, which could model the radial gradient in the concentration of air bubbles in the bubble column (section \ref{sec:hole_ice}) is not considered. Also, a gradient in $z$-direction to model a pressure-caused gradient in the ice properties of the hole ice (also section \ref{sec:hole_ice}) is not included in this study.

The hole-ice cylinders are implemented perfectly aligned along the $z$-direction. A tilt of the cylinders, or even deformations of the cylinders which could be caused by oscillations of the drilling head are not considered in this study.

For the direct-detection acceptance criterion (section \ref{sec:direct_detection}), the optical modules are assumed perfectly aligned along the $z$-direction. Tilted orientations of the optical modules are not accounted for in this study, even though the orientation of the optical modules could be fitted using calibration data. Also, the impact angle is completely ignored when using direct detection, even if transmission or quantum efficiency effects might be direction dependent.

This study also does not provide a ready-to-use geometry definition of the whole detector with all optical-module positions, hole-ice cylinders and cable positions.\footnote{For the current status of the creation of a full geometry file, see \url{https://github.com/fiedl/hole-ice-study/issues/61}.} Also, the performance impact when using such a full geometry file in production simulations is not examined in this study and needs to be evaluated in a follow-up study.\followup

