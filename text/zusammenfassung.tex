\cleardoublepage
\thispagestyle{empty}

\begin{otherlanguage}{ngerman}

\makeatletter
\begin{center}
  \large Der Einfluss von Loch-Eis auf die Ausbreitung und Detektion von Licht im IceCube-Neutrino-Observatorium

  \medskip
  \normalsize \@author

  \@date
\end{center}
\makeatother

\vspace{1cm}

\begin{abstract}

Das \icecube-Neutrino-Observatorium am Südpol verwendet das Eis eines Gletschers als Detektor-Medium, in dem Teilchen aus Neutrino-Reaktionen auf ihrem Weg durch das Eis Licht erzeugen, das von Photo-Detektoren im Eis registriert wird. Loch-Eis (engl. \glqq hole ice\grqq) ist das erneut gefrorene Wasser in den Bohrlöchern, in denen die Detektor-Module in das Eis eingelassen wurden, und hat voraussichtlich andere optische Eigenschaften als das übrige Eis des Gletschers.

Um die Detektor-Kalibrierung und damit die Genauigkeit von \icecube zu verbessern, stellt diese Arbeit neue Algorithmen vor, mit denen die Propagation von Photonen durch Loch-Eis mit verschiedenen Eigenschaften simuliert werden kann. Die Tauglichkeit der Algorithmen wird durch eine Reihe von statistischen Überprüfungen sowie durch einen Vergleich mit Messungen und Simulationen anderer Kalibrierungsstudien untermauert.

Als Anwendungsbeispiele werden in dieser Arbeit die Simulation eines oder mehrerer Loch-Eis-Zylinder mit unterschiedlichen Eigenschaften, etwa der Lage, der Größe, der Absorptions- und der Streulänge des Loch-Eises, die Simulation eines lichtabsorbierenden Kabels sowie die beispielhafte Kalibrierung anhand von Daten des Leuchtdioden-Kalibrierungs-Systems von \icecube durchgeführt.

Die großräumige Ausbreitung von Licht durch das Eis des Gletschers erfolgt nahezu unbeeinflusst von den Eigenschaften des Loch-Eises. Allerdings muss jedes Photon, das von \icecube registriert oder vom Kalibrierungssystem abgegeben wird, zunächst durch das Loch-Eis gelangen, sodass sowohl die Detektor-Einheiten als auch die Kalibrierungs-Leuchtdioden in Abhängigkeit von den Eigenschaften des Loch-Eises effektiv abgeschirmt werden, da ein Teil des Lichtes vom Loch-Eis absorbiert, ein anderer Teil reflektiert werden. Für Detektor-Module, die im Loch-Eis nicht völlig zentrisch zum Liegen gekommen sind, ist dieser Effekt abhängig von der Azimuth-Richtung der Photonen, was sich auch in den Kalibrierungsdaten widerspiegelt.

Aus Kalibrierungsdaten ergibt sich, dass die Detektor-Module richtungsabhängig abgeschirmt werden. Vorläufige Ergebnisse dieser Simulations-Studie zeigen, dass die beobachtete Abschirmung nicht von Kabeln verursacht werden kann, die an der Seite der Detektor-Module verlaufen, sondern die Annahme eines Loch-Eises geeigneter Lage, Größe und optischer Eigenschaften erforderlich ist, um die Kalibrierungsdaten zu erklären.

Die neuen Propagations-Algorithmen werden in Kürze in das Simulations-System von \icecube integriert werden. Studien, die eine geringe Statistik aufweisen, sich also mit Neutrino-Reaktionen befassen, die nur wenig Licht hervorrufen, oder, bei denen das Licht nur von wenigen Detektor-Modulen registriert wird, können mit den in dieser Arbeit vorgestellten Simulationsmethoden systematische Unsicherheiten verringern.

Da die direkte Simulation der Licht-Propagation durch das Loch-Eis jedoch zusätzliche Simulations-Laufzeit erfordert, ist es empfehlenswert, dass Studien mit einer hohen Statistik, die sich also mit Neutrino-Reaktionen befassen, die eine große Menge an Licht hervorrufen, das wiederum von einer Vielzahl von Detektor-Modulen registriert wird, den Einfluss des Loch-Eises durch bereits gebräuchliche Verfahren nur näherungsweise berücksichtigen. Die Parameter dieser Näherungsverfahren sollten jedoch aufgrund der Erkenntnisse aus direkten Simulationen korrigiert werden.

\end{abstract}

\end{otherlanguage}
