%!TEX TS-program = ../make.zsh

\subsection{Angular-Acceptance Simulation For H2 Parameters}
\label{sec:angular_acceptance_simulation_for_h2_parameters}

The so-called H2 hole-ice parameters assume a hole-ice radius of $30\cm$, corresponding to the entire drill hole being filled with hole ice, and a geometric hole-ice scattering length of $50\cm$, corresponding to an effective scattering length of $8.33\m$. \cite{holeicestudieswithyag}

Using the new medium-propagation algorithm (section \ref{sec:algorithm_b}) with direct detection and plane waves as photon sources (section \ref{sec:angular_acceptance_scan}) to generate an effective angular-acceptance curve for these H2 hole-ice parameters:

\hspace{1cm}
\begin{center}
  \includegraphics[width=0.75\textwidth]{img/angular-acceptance-karle-h2-vs-reference}
\end{center}
\hspace{1cm}


The blue a priori curve is based on previous \photonics simulations assuming the same H2 hole-ice parameters. \cite{lundberg, icepaper}

\newpage

The same simulation assuming an effective hole-ice scattering length of $50\cm$, corresponding to a geometric scattering length of $3\cm$:

\hspace{1cm}
\begin{center}
  \includegraphics[width=0.75\textwidth]{img/angular-acceptance-karle-h2-assuming-esca}
\end{center}
\hspace{1cm}

\docpar{These simulations are documented in \issue{80}.}