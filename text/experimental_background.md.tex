%!TEX TS-program = ../make.zsh

# Experimental Background
## IceCube Detector

- cubic-kilometer neutrino detector built into the ice at the South Pole \cite{instrumentation}
- 5160 doms, deployed between 1450m and 2450m below the surface on 86 vertical strings, each consisting of 60 doms deployed along a single cable \cite{instrumentation}
- in-ice array, IceTop, DeepCore \cite{instrumentation}
- vertical spacing 17m, about 125m horizontal separation, hexagonal footprint on a triangular grid \cite{instrumentation}
- energies TeV to PeV scale \cite{instrumentation}

\smallerimage{icecube-schematics-instrumentation} Image source: \cite{instrumentation}

- DeepCore denser instrumented, lower energy threshold. \cite{instrumentation}



## Digital Optical Modules (DOMs)

\smallerimage{dom-components-instrumentation} Image source: \cite{instrumentation}

\smallerimage{dom-schematics-firstyearperformance} Image source: \cite{firstyearperformance}

- basic detection unit in icecube is the digital optical module (DOM) \cite{instrumentation}
- glass sphere, 10 inch PMT, light-emitting diode (LED) flasher board, processing circuitry \cite{instrumentation}

- sensitive to particles 10GeV to 10PeV creating light in the ice, up to 500m away \cite{instrumentation}

- the signals are digitized in the optical sensors to minimize loss of information from degradataion of analog signals sent over long distances \cite{firstyearperformance}

## DOM Angular Acceptance Characteristics
## Angular Acceptance Approximations and Their Limitations
% Outdated model for hole-ice properties
% No different properties for each string/DOMs
% DOM is wrongly considered centred in hole ice
## Likelihood Methods for Comparing Simulations to Reference Data
## Ice Models to Characterize Photon Scattering and Absorption in the South-polar Ice

- timeline of ice models: \url{https://docushare.icecube.wisc.edu/dsweb/Get/Document-79091/ice.pdf} \cite{flasherdataderivedicemodels}
- anisotropic scattering amplitude, aligned with ice flow direction \cite{icrc17pocam}

- air bubble do not cause absorption, only scattering \cite{absorption1997}
- antarctic ice has exceptional optical properties because air bubbles shrink and vanish under large pressure forming an air clathrate hydrate \cite{rongenswedishcamera}

## Hole Ice as Region of Different Ice Properties Around Detector Strings

\smallerimage{swedish-camera-downwards} Image source: \cite{icrc17pocam}

- camera suggests two hole-ice components: clear outer region, central column of small scattering length about 16cm diameter \cite{rongenswedishcamera,instrumentation}
- cylindrical freezing: impurities or air bubbles are pushed along the freezing boundaries until they merge in the center \cite{rongenswedishcamera}

- 68 boreholes, approx 60cm diameter to depth of 250m using hot water drilling\cite{instrumentation}
- instrumentation deployed into the water-filled holes, becoming frozen in place and optically coupled with the surrounding sheet \cite{instrumentation}
- produce hole 48h, freezing about 2weeks \cite{instrumentation}

- camera deployed to monitor the freeze-in process and optical properties of drill hole. Two video cameras in separate spheres, each also equipped with four LEDs and three lasers. \cite{instrumentation}
- drill hole completely refrozen after 15 days \cite{instrumentation}
- clear outer layer and central core of about 16cm diameter with smaller scattering length than bulk ice\cite{instrumentation}
- no long-term changes have been observed \cite{instrumentation}

