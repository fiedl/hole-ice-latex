%!TEX TS-program = ../make.zsh

\cleardoublepage
\thispagestyle{empty}
\begin{otherlanguage}{ngerman}

\makeatletter
\begin{center}
  \large Der Einfluss von Loch-Eis auf die Ausbreitung und Detektion von Licht im IceCube-Neutrino-Observatorium

  \medskip
  \normalsize \@author

  \@date
\end{center}
\makeatother

\vspace{1cm}

\begin{abstract}

Das \icecube-Neutrino-Observatorium am Südpol verwendet das Eis eines Gletschers als Detektor-Medium, in dem Teilchen aus Neutrino-Reaktionen auf ihrem Weg durch das Eis Licht erzeugen, das von Photo-Detektoren im Eis registriert wird.

Um die Detektor-Kalibrierung und damit die Genauigkeit von \icecube zu verbessern, führt diese Arbeit zwei neue Algorithmen ein, mit denen die Propagation von Photonen durch das Loch-Eis (engl. \glqq hole ice\grqq) simuliert werden kann. Loch-Eis ist das erneut gefrorene Wasser in den Bohrlöchern, in denen die Detektor-Module in das Eis eingelassen wurden. Das Loch-Eis kann andere Propagations-Eigenschaften als das umliegende Eis des Gletschers aufweisen, sodass durch dessen Berücksichtigung neue Kalibrierungs-Parameter eingeführt werden.

Die Tauglichkeit der Algorithmen wird durch eine Reihe von Eignungs-Tests und statistischen Überprüfungen sowie durch Gegenüberstellung von Ergebnissen mit denen einer vergleichbaren Studie untermauert.

Als Anwendungsbeispiele werden in dieser Arbeit die Simulation eines oder mehrerer Loch-Eis-Zylinder mit unterschiedlichen Eigenschaften, etwa der Lage, der Größe, der Absorptions- und der Streulänge des Loch-Eises, die Simulation eines lichtabsorbierenden Kabels sowie die beispielhafte Kalibrierung anhand von Daten des Leuchtdioden-Kalibrierungs-Systems von \icecube durchgeführt.

Die großräumige Ausbreitung von Licht durch das Eis des Gletschers erfolgt nahezu unbeeinflusst durch die Eigenschaften des Loch-Eises. Allerdings muss jedes Photon, das von \icecube registriert oder vom Kalibrierungs-System abgegeben wird, zunächst durch das Loch-Eis gelangen, sodass sowohl die Detektor-Module als auch das Kalibrierungs-System in Abhängigkeit von den Eigenschaften des Loch-Eises effektiv abgeschirmt werden, da ein Teil des Lichtes vom Loch-Eis absorbiert, ein anderer Teil reflektiert werden. Für Detektor-Module, die im Loch-Eis nicht völlig zentrisch zum Liegen gekommen sind, ist dieser Effekt abhängig von der Azimuth-Richtung der Photonen, was sich auch in den Kalibrierungsdaten widerspiegelt.

Die Simulation eines lichtabsorbierenden Kabels, das sich an der Seite eines jeden Detektor-Moduls befindet, anstelle eines Loch-Eis-Zylinders zeigt, dass einige, aber nicht alle der vom Kalibrierungs-System beobachteten Effekte von einem solchen Kabel verursacht werden können, sodass ein Loch-Eis mit Eigenschaften, die sich von denen des umliegenden Gletscher-Eises unterscheiden, erforderlich ist, um die Kalibrierungsdaten zu erklären.

Der Einfluss des Loch-Eises auf die Detektion von Licht in den Detektor-Modulen von \icecube kann in Simulationen durch die Verwendung modifizierter Winkelakzeptanz-Charakteristiken der Detektor-Module genähert werden.

Für Studien, die sich mit Neutrino-Reaktionen befassen, die eine große Menge an Licht hervorrufen, das wiederum von einer Vielzahl von Detektor-Modulen registriert wird, ist die Verwendung von modifizierten Winkelakzeptanz-Charakteristiken geeignet, um den Einfluss des Loch-Eises näherungsweise zu beschreiben.

Die derzeit verwendeten Näherungen entsprechen simulierten Loch-Eis-Zylindern mit einem Durchmesser von etwa $30\cm$ und einer effektiven Streulänge von etwa $1\m$. Um ein Loch-Eis mit Eigenschaften zu beschreiben, wie sie von aktuellen Studien gemessen wurden, müssen die derzeit verwendeten durch neue Winkelakzeptanz-Charakteristiken ersetzt werden, die durch Simulationen wie den in dieser Arbeit vorgestellten erstellt werden können.

Für Studien, die eine geringere Statistik aufweisen, sich also mit Neutrino-Reaktionen befassen, die nur wenig Licht hervorrufen, oder, bei denen das Licht nur von wenigen Detektor-Modulen registriert wird, können systematische Unsicherheiten durch die in dieser Arbeit vorgestellten Simulationsmethoden verringert werden. Folgestudien müssen jedoch noch feststellen, wie groß die Verbesserung der Genauigkeit des \icecube-Observatoriums ist, die sich dadurch ergibt.

\end{abstract}

\end{otherlanguage}