%!TEX TS-program = ../make.zsh

\section{Conclusion}
\label{sec:conclusion}

- Key results
- Outlook
- Detector Calibration Can/Cannot (?) Be Improved by Simulating Local Ice Properties Like Hole Ice
- Suggested Follow-up Studies
  - Flasher studies to find hole-ice parameters

- this study
- direct hole-ice proapgation
- nested cylinders
- individual dom displacement relative to hole ice
- cables
- direct detection
- this means more calibration parameters
- and minor effects like effective reflection
- in bulk ice same propagation
- but shielding of the doms for strong hole-ice parameters

- what to do with it
  - individual dom positions
  - individual hole-ice columns: position, radius, absorption length, scattering length
  - nested cylinders
  - cables

- simulations with cable, bc, flasher, to see if hole ice is needed to explain flasher calibration data
- especially if cable can account for h2 model

- effective reflection

- need to quantify effect on detector precision

- in bulk ice same properties, but effective shielding of the doms via hole ice

- current approximation curve represents hole-ice parameters
- when different hole-ice from other studies, e.g. ... from spicehd, new effective angular-acceptance curves need to be produced with \ppc or \clsim
- for high-energy studies, approximation curves can be used
- for low-statistics studies, direct simulation can reduce systematics

