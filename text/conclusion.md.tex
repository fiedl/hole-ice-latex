%!TEX TS-program = ../make.zsh

\section{Conclusion}
\label{sec:conclusion}

This study presented a new method to simulate the propagation of photons through the hole ice around the detector strings of the \icecube neutrino observatory as extension of the standard \clsim photon-propagation-simulation software.

Two algorithms were introduced to accomplish this task.
As a first approach, an algorithm was developed that accounts for the different optical properties of the hole ice by adding subsequent corrections to the calculations of the standard-\clsim algorithm.
In this approach, the changes to the well-tested standard-\clsim code base are kept minimal.
The hole-ice-correction algorithm, however, requires the definition of the hole-ice properties to be relative to the properties of the surrounding ice, which does not allow to implement more current hole-ice models.
Therefore, a second algorithm was devised that handles the transition from bulk ice to hole ice and vice versa the same way it handles other medium transitions like the propagation through ice layers.
Because this approach is more generic, it also allows more complex hole-ice scenarios where a photon crosses more than one hole-ice cylinder between two scatters.
This approach, however, required to rewrite the previous \clsim media-propagation code.
The layer-tilt and anisotropy ice features had to be left out for the time being and need to be reintegrated into the new algorithm, yet.
The validity of the algorithms is supported by unit tests and a series of statistical cross checks.
Also, \clsim hole-ice simulation results were found to agree with simulation results of \ppc, which is separate \icecube simulation software that also supports direct hole-ice propagation, but uses a different approach to implementing the medium transitions.
The performance difference of the standard-\clsim media propagation and the new hole-ice algorithms is smaller than the uncertainties of the performance measurement.
Simulating the direct photon propagation through hole ice with small scattering lengths, however, adds more scattering steps to the simulation, resulting in a longer total simulation time.
The implementation's performance can still be improved by applying more GPU-programming and memory optimizations.

In general, the new method allows to simulate the propagation through hole-ice cylinders with different absorption lengths, scattering lengths, and radii.
The optical modules can be positioned with individual displacements relative to the hole ice.
The method supports the nesting of several hole-ice cylinders of different properties and the simulation of light-absorbing cables.
Optical modules support direct detection and thereby can accept photon hits based on whether the photon hit the sensitive area of the optical module rather than based on the impact angle.
These features allow to fit additional calibration parameters such as the hole-ice positions, the hole-ice radius and the hole-ice scattering length.

Direct-photon-propagation simulations indicate that light propagation through the bulk ice of the \icecube detector is largely unaffected by the hole ice. But each photon that is detected by the optical modules, and every photon that is emitted by the calibration LEDs, needs to propagate through the hole ice and is affected by the properties of the hole ice.
Depending on the hole-ice properties, the optical modules are effectively shielded by the hole ice. A fraction of the photons is absorbed during the random walk through the hole ice, or effectively reflected by the hole ice.
Evaluating calibration data indicates a strongly asymmetric shielding of the detector modules. Preliminary flasher simulations with direct photon propagation hint that this cannot be accounted for by the shadow of cables, but can be explained by simulating hole ice with a suitable scattering length, size and position relative to the detector modules.
Considering these effects and the new simulation capabilities, follow-up studies should fit the hole-ice scattering length and the hole-ice size and position relative to the positions of the optical modules to improve detector calibration.

The new simulation method will be integrated into \icecube's simulation framework.
Low-statistics studies can use the new method to propagate all simulated photons in order to directly reduce systematic uncertainties.
As the direct hole-ice propagation requires additional simulation time, high-statistics studies should continue to use approximation techniques, which modify the angular-acceptance behavior of the simulated optical modules to effectively account for the average effect of the propagation through the hole-ice cylinders.
By comparing the current standard hole-ice approximation to direct simulations, both methods were found to disagree.
When simulating hole ice with the properties that have been assumed when creating the hole-ice-approximation curves, the new simulations indicate a less pronounced hole ice-effect than the current approximations.
As some, but not all of the recent calibration studies also suggest a weaker hole ice, there is no obvious conclusion on which hole-ice model is to be preferred for the time being.
With the new simulation tools, however, more detailed hole-ice-simulation studies can be performed aiming to find a hole-ice description that is able to reproduce all calibration observations.
The approximation curves can then be updated according to direct-simulation results.

