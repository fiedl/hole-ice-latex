\subsection{Hole Ice Trajectory
Corrections}\label{hole-ice-trajectory-corrections}

Consider a random photon trajectory through the ice. The photon is
created by a particle interaction at the starting point of the
trajectory. On its way, the photon is scattered several times within the
ice before it is absorbed and thereby detected in one of IceCube's DOMs.

\image{photon-trajectory-oheeL3ai.pdf}

\begin{itemize}
\item
  Consider path of photon through ice.
\item
  Interaction propability determined by ice properties and photon
  wavelength.
\item
  Interaction points determined by these properties and random process.
\item
  Look at part of this trajectory, A B between two interactions.
\item
  Change scenario by adding hole ice cylinder.
\item
  Determine how the trajectory is affected by this.
\item
  Consider a photon trajectory through the ice that partially goes
  through hole ice.
\item
  Within hole ice, the ice properties are different from outside.
\item
  Most notably, the impurities (dirt) cause the photons to scatter or be
  absorbed more often there.
\item
  I.e. Scattering and absorption coefficients are larger within hole
  ice.
\item
  Correspondingly, the scattering and absorption lengths, which are the
  mean free distances until scattering or absorption, are shorter.
\end{itemize}

\image{photon-trajectory-aiph6ahD.pdf}

\begin{itemize}
\tightlist
\item
  Consider simulation of a photon.
\item
  \(A\) is the last point of interaction, i.e.~fixed.
\item
  \(B\) is determined by random process.
\item
  Consider a scenario where a photon trajectory starts at point \(A\)
  and ends at point \(B\), where it is scattered or absorbed. The photon
  does not interact inbetween \(A\) and \(B\).
\end{itemize}

\image{photon-trajectory-Edahi9sh.pdf}

\begin{itemize}
\tightlist
\item
  Now, change scenario: add hole ice cylinder.
\item
  Due to locally increased interaction coefficients, the free path is
  shorter, i.e.~modified \(B\).
\item
  Why is the path shorter?
\item
  B is determined by random variable.
\item
  If the mean free path is shorter (because the likelyhood if
  interaction is larger), then the random variable is smaller.
\item
  The change is \(\Delta b:= \overline{AB'} - \overline{AB}\).
\end{itemize}
