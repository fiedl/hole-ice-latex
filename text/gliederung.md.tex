%!TEX TS-program = ../make.zsh

## Implementing Hole-Ice Propagation as Correction of the Existing Propagation Algorithm
% Algorithmen A+B:
% motivation
% pros + cons
% flow charts
% context: task, before, after, input, output
% mention intersection, projection and termination point where needed and refer to repo or appendix.

## Implementing Hole-Ice Propagation as Generic Medium Change by Rewriting the Existing Propagation Algorithm
% Algorithmen A+B:
% motivation
% pros + cons
% flow charts
% context: task, before, after, input, output

## Technical Issues and Optimizations

## Unit Tests and Consistency Checks
### Unit Tests With Single Photons
### Instant Absorption Tests
### Arrival Time Distributions
### Absorption Length Distributions

# Application Examples
## Simulating a Hole-Ice Column With Distinct Scattering Properties
## Scanning the Angular Acceptance of an Optical Module
% Plane-Wave vs. Pencil-Beam
% Direct Detection
## Parameter Grid Scan to Find Hole-ice Properties Compared to Reference Simulation / Determining the Hole-Ice Parameters of the Current Hole-Ice Approximation
## Simulating Nested Hole-Ice Columns
## Simulating the Displacement of Optical Modules Relative to the Hole-Ice Columns
## Simulating Shadowing Cables as Opaque Cylinders
## Scanning Hole-Ice Parameters for the Best Agreement with Flasher Calibration Data

# Discussion
% What do the results mean?
% Why is it interesting?
% What does it change?

## Improvements for Detector Callibration

## Comparison to Other Hole-Ice-Propagation Software
% ppc
% approximation with standard clsim
% inkl. performance-vergleich

## Performance Considerations
% taugt für low-energy-studies

## Ice Features Not Considered in this Study
### Tilted DOM Positions
### Gradient in Hole-ice Properties

# Conclusion
% Key results
% Outlook
## Detector Calibration Can/Cannot (?) Be Improved by Simulating Local Ice Properties Like Hole Ice
## Suggested Follow-up Studies
% Flasher studies to find hole-ice parameters
