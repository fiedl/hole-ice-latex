%!TEX TS-program = ../make.zsh

\tableofcontents

# Introduction
% What is the work?
% Why is it important?
% What is needed to understand the work?
% How will the work be presented?
% Definitions and key terms

## What is IceCube
## IceCube as Particle Physics Laboratory
## IceCube as Deep-space Neutrino Telescope
## Improving the Detector Resolution Through Calibration
## Can Local Ice Properties be Implemented in Simulations Aiming to Improve Calibration?

# Theoretical Background
## Neutrinos
## Relevant Neutrino Reactions
## Photon Model
## Scattering Model

# Experimental Background
## IceCube Detector
## Digital Optical Modules (DOMs)
## DOM Angular Acceptance Characteristics
## Event Reconstruction
## Likelihood Methods for Comparing Simulations to Reference Data
## Ice Models to Characterize Photon Scattering And Absorption
## Hole Ice As Region with Different Ice Properties Around Detector Strings

# Simulation Background
## Parallal Computing
## Photon Propagation Algorithm
## `clsim` Photon Propagator

# Methods
% Everything needed to reproduce the work
## Photon Propagation Through Different Media
### Principles of Adding Medium Changes to Photon Propagation
### Adding Hole Ice as Post-propagation-step Correction
### Adding Hole Ice to Generic Medium Changes
### Unit Testing
## Implementing Cylinder-shaped Areas with Distinct Ice Properties
### Hole-ice Cylinders
### Nested Cylinders With Different Properties
### Cables With Instant Absorption
## Example Studies
### Basic Scattering Properties
### Instant Absorption
### Shifting the DOMs Relative to the Hole-ice Cylinder
### Cable Shadows
## Parameter Grid Scan to Find Hole-ice Properties
### Based on Reference Simulation
### Based on Flasher Data ($\rightarrow$ out of scope)
## Full-cycle simulation to Check for Sensitivity to Simulated Features ($\rightarrow$ out of scope)

# Results
% All results relevant to the question
## Impact of Simulated Features on the Effective DOM Angular Acceptance
### Effect of Hole-ice Cylinders
### Effect of Nested Cylinders
### Effect of Cable Shadows
### Effect of Shifted DOM Positions
## Estimated Hole-ice Parameters of Reference Simulation
## Impact of Simulated Features on Event Reconstruction ($\rightarrow$ out of scope)

# Discussion
% What do the results mean?
% Why is it interesting?
% What does it change?
## Simulation Sensitivity to Local Ice Features
## Estimated Gain on Reconstruction Resolution ($\rightarrow$ out of scope)
## Performance Impact on Simulation
## Estimated Uncertainties Due to Numerical Limitations
## New Features are Most Useful in Low-energy Studies (?)
## Comparison to Alterantive Propagation Tool, \textit{Photon Propagation Code}, `ppc`
## Features Not Considered in this Simulation
### Tilted DOM Positions
### Gradient in Hole-ice Properties

# Conclusion
% Key results
% Outlook
## Detector Calibration Can/Cannot (?) Be Improved By Simulating Local Ice Properties Like Hole Ice
## Suggested Follow-up Studies

