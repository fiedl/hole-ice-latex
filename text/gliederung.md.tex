%!TEX TS-program = ../make.zsh

\setcounter{tocdepth}{2}
\tableofcontents

# Introduction
% What is the work?
% Why is it important?
% What is needed to understand the work?
% How will the work be presented?
% Definitions and key terms

## What is IceCube
## IceCube as Particle Physics Laboratory
## IceCube as Deep-space Neutrino Telescope
## Improving the Detector Resolution Through Calibration
## Can Local Ice Properties be Implemented in Simulations Aiming to Improve Calibration?

# Theoretical Background
## Neutrinos
## Neutrino Interactions Relevant to IceCube
## Cherenkov-light Emission
## Photon Scattering Model

# Experimental Background
## IceCube Detector
## Digital Optical Modules (DOMs)
## DOM Angular Acceptance Characteristics
## Angular Acceptance Approximations and Their Limitations
% Outdated model for hole-ice properties
% No different properties for each string/DOMs
% DOM is wrongly considered centred in hole ice
## Event Reconstruction
## Likelihood Methods for Comparing Simulations to Reference Data
## Ice Models to Characterize Photon Scattering and Absorption in the South-polar Ice
## Hole Ice as Region of Different Ice Properties Around Detector Strings

# Simulation Background
## Monte-carlo Simulations
## Parallal Computing on Graphics Processing Units (GPUs)
## Photon Propagation Algorithm
## `clsim` Photon Propagation Simulation

# Methods
% Everything needed to reproduce the work
## Photon Propagation Through Different Media
### Principles of Adding Medium Changes to Photon Propagation
### Adding Hole Ice as Post-propagation-step Correction
### Adding Hole Ice to Generic Medium Changes
### Unit Testing
## Implementing Cylinder-shaped Areas with Distinct Ice Properties
### Hole-ice Cylinders
### Nested Cylinders with Different Properties
### Cables with Instant Absorption
## Example Studies
### Basic Scattering Properties
### Instant Absorption
### Shifting the DOMs Relative to the Hole-ice Cylinder
### Cable Shadows
## Parameter Grid Scan to Find Hole-ice Properties Compared to Reference Simulation
## Comparison of Flasher Data and Simulated Flasher Events
% ## Full-cycle Simulation to Check for Sensitivity to Simulated Features (→ outofscope)

# Results
% All results relevant to the question
## Impact of Simulated Features on the Effective DOM Angular Acceptance
### Effect of Hole-ice Cylinders
### Effect of Nested Cylinders
### Effect of Cable Shadows
### Effect of Shifted DOM Positions
## Estimated Hole-ice Parameters of Reference Simulation
## Estimated Hole-Ice Parameters Based on Flasher Data
% ## Impact of Simulated Features on Event Reconstruction ($\rightarrow$ out of scope)

# Discussion
% What do the results mean?
% Why is it interesting?
% What does it change?
## Estimated Detection Sensitivity to Local Ice Features in Simulations
## Estimated Gain on Reconstruction Resolution ($\rightarrow$ out of scope ?)
## Performance Impact on Simulations
## Estimated Uncertainties Due to Numerical Limitations
## New Simulation Features are Most Useful in Low-energy Studies (?)
## Comparison to Alterantive Propagation Tool, \textit{Photon Propagation Code}, `ppc`
## Ice Features Not Considered in this Simulation
### Tilted DOM Positions
### Gradient in Hole-ice Properties

# Conclusion
% Key results
% Outlook
## Detector Calibration Can/Cannot (?) Be Improved by Simulating Local Ice Properties Like Hole Ice
## Suggested Follow-up Studies
% Flasher studies to find hole-ice parameters
