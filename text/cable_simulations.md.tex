%!TEX TS-program = ../make.zsh

\subsection{Simulating Shadowing Cables as Opaque Cylinders}
\label{sec:cables}

- on slack, martin has argued why noone could kill the h2 for very long

> mrongen [7:53 PM 2018-03-26]
> Ok getting back to the bigger picture:
> The overall state of observations IMHO only leaves us with three plausible true realizations:
>  1. There is no bubble column
>  2. It is (usually) small and centered around the DOM
>  3. It is (usually) at the cable position
>
> These hypothesis (no bubble column being the null hypothesis against which the other two are tested) are actually fairly easy to test given the usual machinery. Use the new PPC with photon deletion on the cable, use direct detection (is that still in there?) and scan scattering length, size for both geometrical hypothesis (3 might need a small adjustment in PPC...) if the LLH improvement is significant that's it.
>
> A cross check for hypothesis 1 is also to see which scattering length (in a simulation without the cable and the column filling the entire drill hole) fits simulation data with the cable only. If it is 50cm that kind of explains why H2 has been so hard to kill (edited)


\url{https://github.com/fiedl/hole-ice-study/issues/101}


\cite{martinspicehddard}, slide 39:
> Currentcableimplementationdeletesphotonfrom
> a given range of azimuthal impact directions
> → cable is not allowed to be inside bubble