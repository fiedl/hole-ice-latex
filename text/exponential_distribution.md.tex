%!TEX root = ../diplomarbeit.tex

\subsection{Exponential Distribution of the Total Photon Path Length}
\label{sec:exponential_distribution}

This section shows why the total-path-length distribution of photons propagating through a homogeneous medium described in section \ref{sec:total_path_length_distribution} is expected to follow an exponential curve.

% % http://pgfplots.sourceforge.net/pgfplots.pdf
% \begin{tikzpicture}
%   \begin{axis}
%     \addplot{e^(-x)};
%   \end{axis}
% \end{tikzpicture}

Let $N$ be the total number of started within the medium. Let $\lambda\abs$ be the photon absorption length within the medium.

The probability $p(x):=f\,\dx$ for a photon to be absorbed within its path length interval $[x; x + \dx[$ is the same as long as the photon stays within the homogeneius medium.

Let $n(x)$ be the number of photons with a path length of $x$ or more, that is to say the number of photons that still exist with a path length greater or equal $x$.

The number of photons $m(x)$ that are absorbed within the interval $[x; x + \dx[$ is determined by the change of the number of remaining photons, which in the limit of many photons is proprtional to the absorption probability $p(x)$.

$$ m(x) = -\dn(x) = p(x)\ n(x) = f\,\dx\,n(x) $$

$$ \frac{\d}{\dx}\ n(x) = -f\ n(x) $$

As the derivative of $n$ is proportional to $n$, $n$ is an exponential.

$$
  n(x) = a\,\e^{b\,x}, \ \ \
  \frac{\d}{\dx}\,n(x) = b\,a\,\e^{b\,x} = b\,n(x) = -f\,n(x)
$$

% Comparing the coefficients, find the relation of $f$ and $b$:
%
% $$
%   \frac{\d}{\dx}\ n(x)
%   = \frac{\d}{\dx}\ a\,\e^{b\,x}
%   = a\,b\,\e^{b\,x}
%   = -f\ n(x)
%   = -f\,a\,\e^{b\,x}
%   \ \
%   \Rightarrow
%   \ \
%   b = -f
% $$

The absorption length $\lambda\abs$ is defined as the distance after that the number of photons has dropped to $\sfrac{1}{\e}$ of the original number.

$$
  n(x) = a\,\e^{b\,x}, \ \
  n(\lambda\abs) = \frac{1}{\e}\, n(0)
  \ \ \Rightarrow \ \
  b = \sfrac{-1}{\lambdaabs}
$$

% absorption probability p at each point.
% p = dx f
% probability to be absorbed after distance x: P(x) is proportional to number of pho
% the number of photon with a total path length of x is proportional to the total number of photons and the probability of a photon to have this path length
% probability of a photon having a path length of x: depends on the absorption length or the absorption probability at each point
% \int_0^x{dx f}

$$
  n(x) = n(0) \cdot \e^{\sfrac{-x}{\lambda\abs}}, \ \ \ n(0) = N
$$

% Falsch [n(x) ist die Anzahl von Photonen, die noch da sind.]: The number of photons $n(x)$ with total path length $x$, $\lambda_\text{abs}$ is the absorption length within the hole-ice cylinder. $N$ is the total number of photons started in the simulation.
%
% $$
%   n(x) \propto N \cdot \e^{-\frac{x}{\lambda_\text{abs}}}
% $$

In a histogram of the total path lengths $X$, the bin height is proportional to the number $m(x)$ of photons that are absorbed within the interval $[x; x + \dx[$, not the number $n(x)$ of remaining photons.

$$
  m(x) = p(x)\,n(x) = p(x)\,N\,\e^{\sfrac{-x}{\lambda\abs}}
$$

In the case of a homogeneous medium where $p(x)$ is constant for all $x$, $p(x) = p_0$, the histogram should also follow an exponential curve.

$$
  m(x) = p_0\,N\,\e^{\sfrac{-x}{\lambda\abs}}
$$

From the rate of the exponential decay, one can read the absorption length $\lambda\abs$.

If, on the other hand, there is a medium border at $x_0$ such that the absorption probability is piecewise defined,

$$
  p(x) = \begin{cases}
    p_0 & : x \leq x_0 \\
    p_1 & : x > x_0
  \end{cases}
$$

then the histogram follows a piecewise defined curve, consisting of two exponential curves.

$$
  m(x) = \begin{cases}
    p_0\,N\,\e^{\sfrac{-x}{\lambda_0}} & : x \leq x_0 \\
    p_1\,N\,\e^{\sfrac{-x}{\lambda_1}} & : x > x_0
  \end{cases}
$$

From the rates of the exponential decays, one can read the absorption lengths $\lambda_0$ and $\lambda_1$ of both media from the histogram.

Note that as the histogram shows the number $m(x)$ of decayed photons within an interval, not the number $n(x)$ of remaining photons, the curve the histogram follows does not need to be contiuous but may have a jump discontinuity at the position $x_0$ of the medium border.

The number $n(x)$ of remaining photons, however, is continuous, also at the position $x_0$ of the medium border.

$$
  n(x) = N - \int_0^x m(x) = \begin{cases}
    N - f_0\,N\,\lambda_0\,\e^{\sfrac{-x}{\lambda_0}} & : x \leq x_0 \\
    n(x_0) - f_1\,N\,\lambda_1\,\e^{\sfrac{-x}{\lambda_1}} & : x > x_0
  \end{cases}
$$
