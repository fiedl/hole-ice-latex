%!TEX TS-program = ../make.zsh

\subsection{Performance Considerations}
\label{sec:performance}

The performance of the different propagation algorithms, that is to say the time it takes to simulate a certain number of propagation steps with each algorithm, is about the same. The number of propagation steps, however, which is determined by the number of scatterings, has a significant effect on the total simulation time.

Figure \ref{fig:Go7Maquo} shows a comparison of the standard-\clsim algorithm (section \ref{sec:standard_clsim}), the hole-ice-correction algorithm (section \ref{sec:algorithm_a}), and the new medium-propagation algorithm with hole ice (section \ref{sec:algorithm_b}), each running a simulation propagating $10^5$ photons on a CPU.\footnote{Test system configuration for CPU simulation: OS: macOS Sierra 10.12.6 16G1212 x86\_64, Kernel: 16.7.0, CPU: Intel i7-4870HQ (8) @ 2.50GHz, GPU: Intel Iris Pro, NVIDIA GeForce GT 750M, RAM: 16384MiB.}

\begin{figure}[htbp]
  \image{performance-comparison}
  \caption{Performance comparison of the of the standard-\clsim algorithm (section \ref{sec:standard_clsim}), the hole-ice-correction algorithm (section \ref{sec:algorithm_a}), and the new medium-propagation algorithm with hole ice (section \ref{sec:algorithm_b}). The chart shows the total propagation time of $10^5$ photons for these scenarios: From top (slowest) to bottom (fastest): (1) New medium-propagation algorithm with strong hole ice. (2) Hole-ice-correction algorithm with strong hole ice. (3) New medium-propagation algorithm with moderate hole ice, without ice layers. (4) Standard \clsim. (5) New medium-propagation algorithm without hole ice and without ice layers. (6) New medium-propagation algorithm with moderate hole ice, without ice layers. (7) New medium-propagation algorithm without hole ice, but with ice layers. (8) Hole-ice-correction algorithm where the scattering length inside the hole ice is the same as in the bulk ice. For (3)-(8), the performance measured time difference is smaller than the statistical uncertainty of the time measurement. Increasing the number of scattering steps, however, which corresponds to the measurements (1) and (2), shows an increased simulation time.}
  \label{fig:Go7Maquo}
\end{figure}

\docpar{The implementation of this performance measurement is documented in \issue{49}.}

The difference of the total simulation time for each algorithm is smaller than the statistical uncertainty of the time measurement as long as the scattering length of the hole-ice cylinder is kept moderate. When moving to smaller scattering lengths within the hole-ice, however, the total simulation time increases, as the total number of simulation steps increases, because one simulation step propagates the photon from one scattering point to the next scattering point.

For a fixed hole-ice-cylinder radius of $30\cm$, going from a geometric hole-ice scattering length of $20\m$ to $0.005\m$ increases CPU propagation time per photon by about $15\,\%$.

% Performance comparison in flasher simulations:
% 2018-07-16
%
%     [2018-07-16 15:27:14] fiedl@kepler00 /afs/ifh.de/group/amanda/scratch/fiedl/hole-ice-study/scripts/FlasherSimulation
%     ▶ ./run.rb --width=127 --brightness=127 --hole-ice=approximation
%
%     [2018-07-16 14:58:38] fiedl@kepler00 /afs/ifh.de/group/amanda/scratch/fiedl/hole-ice-study/scripts/FlasherSimulation
%     ▶ ./run.rb --width=127 --brightness=127 --thinning-factor=0.1
%
% Results:
%
% * Standard clsim with 1/10 thinning: 11min
% * New medium-propagation algorithm
%     without hole-ice cylinders, with 1/10 thinnng: 10min
% * New medium-propagation algorithm
%     with sca_hi = 1/10 sca_bulk,
%     with 1/10 thinning: 15min
%
% kepler00:
%   OS: Scientific Linux release 7.4 (Nitrogen) x86_64
%   Kernel: 3.10.0-862.6.3.el7.x86_64
%   CPU: Intel Xeon E5-2660 0 (32) @ 3.000GHz
%   GPU: NVIDIA Tesla K20m
%   RAM: 2361MiB / 64215MiB

The same effect can be observed when comparing the run time of flasher simulations on a GPU.\footnote{Test system configuration for GPU simulation: Scientific Linux release 7.4 (Nitrogen) x86\_64, Kernel: 3.10.0-862.6.3.el7.x86\_64, CPU: Intel Xeon E5-2660 0 (32) @ 3.000GHz, GPU: NVIDIA Tesla K20m, RAM: 64215 MiB}
Running a flasher simulation with standard \clsim using hole-ice approximation takes about 11 minutes. Running the same simulation with the same number of photons, with the new medium-propagation algorithm, but without any hole-ice cylinders takes about 10 minutes. Running the same simulation, but adding hole-ice cylinders with $36\cm$ radius and a scattering length of $\sfrac{1}{10}$ of the surrounding bulk ice takes about 15 minutes, that is to say, increases the simulation time by about $50\,\%$.

As the number of simulation steps is the dominant factor for the required simulation time, it is desirable to further optimize the implementation of the simulation step (section \ref{sec:technical_issues_and_optimizations}), because even small improvements, multiplied by the number increased number of scattering steps for hole-ice simulations, cause a considerable total performance gain.\followup

These performance considerations also lead to the question when it is most useful to use hole-ice propagation algorithms and when to use the default hole-ice approximation with effective angular-sensitivity curves.

The approximation does not consider the effective reflection of photons when entering the hole ice (section \ref{sec:scattering_simulation}), but only removes a certain percentage of photons for each incoming angle. Also, the approximation considers each optical module perfectly centered within the hole ice rather than accounting for the individual displacement of each module.

In high-energy events with large statistics, that is to say many photons, and where many different optical modules are involved, the displacement effects are expected to cancel out and the reflection effects are expected to be negligible. For these scenarios, in particular for events with long high-energy muon tracks, the hole-ice approximation is expected to be sufficient.

For lower-energy events and events where only a few optical modules are involved, using a calibrated direct photon propagation simulation is expected to reduce simulation systematics. Further studies, however, are required to quantify this reduction of systematics, or equivalently to quantify the gain in precision.\followup

