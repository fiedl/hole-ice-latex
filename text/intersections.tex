%!TEX root = ../diplomarbeit.tex

\subsection{Calculating intersections of photon trajectories with hole-ice cylinders}

In order to make the hole-ice simulation more efficient, one needs to calculate the intersection points of the photon trajectories with the hole-ice cylinders.

\image{intersection-Kahm4UeY.pdf}

\subsubsection{Analytic approach}

Consider the starting point $A := (A_x, A_y)$ and the ending point $B := (B_x, B_y)$ of the trajectory.

The length $\len{AB}$ of the trajectory is given by the norm $\norm{.}$ of the vector difference $\vec{AB}$ of starting and ending point.
\begin{equation} \notag
  \len{AB} = \norm{\vec{AB}}, \ \ \vec{AB} \identical \vec{B} - \vec{A}, \ \ \norm{\vec{v}} \identical \sqrt{v_x^2 + v_y^2}
\end{equation}

In order to find the intersection points $X_1$ and $X_2$, solve the vectorial system of equations

\begin{equation}
  \vec{A} + s \, \vec{AB} = \vec{M} + \vec{MX}
\end{equation}
\begin{equation}
  \norm{\vec{MX}} = r
\end{equation}

for the scalar parameter $s$. The equation system is quadratic in $s$ resulting in zero, one or two solutions.

\begin{equation}
  s_{1,2} = \frac{-\beta \mp \sqrt{\beta^2 - 4\alpha\gamma}}{2\alpha}
\end{equation}
\begin{equation}
  \alpha = (B_x - A_x)^2 + (B_y - A_y)^2
\end{equation}
\begin{equation}
  \beta = 2\,A_x(B_x-A_x) + 2\,A_y(B_y-Ay) - 2\,M_x(B_x-A_x) - 2\,M_y(B_y-A_y)
\end{equation}
\begin{equation}
  \gamma = A_x^2 - 2\,A_x\,M_x^2 + M_x^2 + A_y^2 - 2\,A_y\,M_y + M_y^2 - r^2
\end{equation}

The term under the square root is also called \textbf{discriminant} $D$.
\begin{equation}
  D = \beta^2 - 4\alpha\gamma
\end{equation}
For $D < 0$, the square root does not exist in $\reals$ and therefore, no intersection point exists. For $D = 0$, there is only one intersection point, which is a tangent point. For $D > 0$, there are two intersection points.

Note that $s = 0$ at the starting point $A$, $s = 1$ at the ending point $B$, $s = s_1$ at the first intersection point $X_1$ and $s = s_2$ at the second intersection point $X_2$.

Therefore, the intersection point coordinate vectors $\vec{X_1}$ and $\vec{X_2}$ are given by:
\begin{equation}
  \vec{X_1} = \vec{A} + s_1 \, \vec{AB}
\end{equation}
\begin{equation}
  \vec{X_2} = \vec{A} + s_2 \, \vec{AB}
\end{equation}

How this result can be verified using a symbolic mathematics library, is shown in \appendixref{app:calculating_intersections_with_sympy}.



\subsubsection{TODO}

- \todo{how many jumps? this is a factor for:}
- \todo{How much does it add to the simulation time?}

