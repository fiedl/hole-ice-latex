%!TEX TS-program = ../make.zsh

\section{Examples of Application}
\label{sec:applications}

This section shows examples of what can be done with the new hole-ice algorithms. Section \ref{sec:scattering_simulation} visualizes the scattering of photons within hole-ice cylinders with different scattering lengths. Sections \ref{sec:angular_acceptance_scan} and \ref{sec:parameter_scan} demonstrate scanning the effective angular acceptance of optical modules for different hole-ice parameters. Sections \ref{sec:cylinder_shift} to \ref{sec:cables} show examples of shifting and nesting hole-ice cylinders as well as adding cables as cylinders of instant absorption. Section \ref{sec:flasher} shows an example of a flasher-calibration study.


\subsection{Visualizing a Hole-Ice Column With Different Scattering Lengths}
\label{sec:scattering_simulation}

One of the first things one can use the new propagation algorithms for, is to visualize the scattering behaviour of photons entering a hole-ice cylinder.

\paragraph{Simulation}
In this example, 100 simulated photons are propagated towards and into a hole-ice cylinder. The photons are started from random positions within a 1m-by-1m plane in a distance of $1\m$ from the cylinder center parallely towards the cylinder. (See figure \ref{fig:Uo8kuo2z} a.)

The cylinder radius is $30\cm$. The scattering length within the bulk ice is $\lambda\sca = 1.3\m$, corresponding to an effective scattering length of $\lambda\esca = 21.7\m$. The absorption length within the bulk ice is $\lambda\abs = 48.0\m$.

The scattering length $\lambda\hi\sca$ within the hole ice is specified relative to the bulk-ice scattering length: $\lambda\hi\sca = f\,\lambda\sca, \ f \in \reals_0^+$, for example $f = \sfrac{1}{10}$ in figure \ref{fig:Uo8kuo2z} (c), $f = \sfrac{1}{100}$ in figure \ref{fig:Uo8kuo2z} (d), and $f = \sfrac{1}{1000}$ in figure \ref{fig:Uo8kuo2z} (d).

This simulation uses the hole-ice-correction algorithm described in section \ref{sec:algorithm_a}. The interaction with the optical detector module is deactivated in this simulation.

\docpar{The configuration and implementation of this simulation is documented in \issue{40}.}

\sourcepar{A script for configuring and running this and other simulations of this kind is provided at \url{https://github.com/fiedl/hole-ice-study/tree/master/scripts/FiringRange}.}

\paragraph{Visualization}
The photon-propagation simulation records the starting point, each scattering point, and the final trajectory point of each photon. These photon trajectories can be visualized using the \noun{Steamshovel} event display software.

Photon trajectories are represented as lines connecting the scattering points. The colors of the trajectory segments indicate simulation steps. A red trajectory segment represents a photon that has just been created. The blue end of the spectrum represents a photon that is about to be absorbed.

Figure \ref{fig:Uo8kuo2z} shows the \noun{Steamshovel} visualizations of the simulation described above for different hole-ice scattering lengths.

\youtubepar{An animated visualization of this photon-propagation simulation can be found at\newline \url{https://youtu.be/BhJ6F3B-I1s}.}

\begin{figure}[htbp]
  \subcaptionbox{$\lambda\hi\sca = 1.0 \cdot \lambda\sca$, 3D view. The hole-ice is represented by the grey cylinder around the optical module represented by a white circle.}{\halfcropimage{example-Uo8kuo2z-sca1-0-steamshovel-color-3d}{0 0 0 2.2cm}}\hfill
  \subcaptionbox{$\lambda\hi\sca = 1.0 \cdot \lambda\sca$, view from above}{\halfimage{example-Uo8kuo2z-sca1-0-steamshovel-color}}\hfill
  \subcaptionbox{$\lambda\hi\sca = 0.1 \cdot \lambda\sca$}{\halfimage{example-Uo8kuo2z-sca0-1-steamshovel-color}}\hfill
  \subcaptionbox{$\lambda\hi\sca = 0.01 \cdot \lambda\sca$}{\halfimage{example-Uo8kuo2z-sca0-01-steamshovel-color}}\hfill
  \subcaptionbox{$\lambda\hi\sca = 0.001 \cdot \lambda\sca$}{\halfimage{example-Uo8kuo2z-sca0-001-steamshovel-color}}\hfill
  \subcaptionbox{$\lambda\hi\sca = 0.0001 \cdot \lambda\sca$}{\halfimage{example-Uo8kuo2z-sca0-0001-steamshovel-color}}\hfill
  \subcaptionbox{$\lambda\hi\sca = 0.0001 \cdot \lambda\sca$, zoom}{\halfimage{example-Uo8kuo2z-sca0-0001-steamshovel-color-zoom}}\hfill
  \subcaptionbox{$\lambda\hi\sca = 0.0001 \cdot \lambda\sca$, zoom 2}{\halfimage{example-Uo8kuo2z-sca0-0001-steamshovel-color-zoom2}}\hfill
  \caption{\noun{Steamshovel} visualization of a photon-propagation simulation where photons are started from a plane on the right hand side onto a hole-ice cylinder with a radius of $30\cm$ from $1\m$ distance. The scattering length $\lambda\sca\hi$ within the hole-ice cylinder is given relative to the scattering length $\lambda\sca$ of the surrounding bulk ice. The absorption length is kept the same within the cylinder as in the bulk ice. The geometric scattering length of the bulk ice is $\lambda\sca = 1.3\m$, the absorption length is $\lambda\abs = 48.0\m$. Colors of the photon trajectories indicate simulation steps, that is to say the number of scatterings relative to the total number of scatterings of the photon until absorption. Red: photon just created, blue: photon about to be absorbed. The optical module is shown as a white sphere only to indicate the scale of the scenary. Interaction with the optical module is turned off in the simulation. Animation on \noun{YouTube}: \protect\url{https://youtu.be/BhJ6F3B-I1s}}
  \label{fig:Uo8kuo2z}
\end{figure}

\paragraph{Observations}
If the scattering length of the hole-ice is the same as the scattering length of the bulk ice (figure \ref{fig:Uo8kuo2z} a and b), the photons pass right through the cylinder as if it were not there.

The mean scattering angle in the bulk ice as well as in the hole ice is assumed as \ang{20} \cite{escawiki}. Therefore, for large scattering lengths, that is to say for weak hole ice, the photons are deflected within the hole ice. See figure \ref{fig:Uo8kuo2z} (c) in comparison to (b).

For smaller scattering lengths, that is to say stronger hole ice, several scatterings occur within the hole ice, changing the direction of the scattered photons more drastically. This makes the hole-ice cylinder cause an effective reflection of the incoming photons. See figure \ref{fig:Uo8kuo2z} (d).

In figure \ref{fig:Uo8kuo2z} (e), the scattering length is so small that the space between the cylinder border and the optical module is large enough to show the typical ``random walk'' of the randomly scattering photons. This random scattering delays the photons on their way to the optical module (compare section \ref{sec:arrival_time} regarding arrival-time distributions).

For even stronger hole ice, which is shown in figures \ref{fig:Uo8kuo2z} (f), (g), and (h), the hole-ice cylinder around the optical module is shielding the module from the incoming photons. The small scattering length confines the random walk of the photons to the outer region of the cylinder as it is unlikely for a photon to be propagated further into the cylinder. The photons walk in the outer region until they are absorbed, or scattered out of the cylinder and escape.

\todo{Describe abs-len-about-5cm-sca-0-001-steamshovel to demonstrate the absorption length. No photon can come closer than 5cm from the border.}