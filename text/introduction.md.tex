%!TEX TS-program = ../make.zsh

\section{Introduction}
\label{sec:intro}

- icecube is a neutrino observatory located at earth's south pole
- it uses a cubic kilometer of glacial ice as detector medium \cite{evidence2013}
- where secondary particles from neutrino interactions produce light as they move through the ice
- which is then detected by an array of photo detectors that are deployed throughout the ice.

- better understanding of the ice reduces systematic effects significally \cite{icrc17pocam}
- systematic uncertainty in IceCube coming from ice properties is in the order of 10\%, dominant factor for a number of analyses \cite{icrc17pocam}
- each photon detected by an optical module needs to travel through the refrozen water of the drill hole, called hole ice, properties less known, one of the largest uncertainties for neutrino oscillation \cite{icrc17pocam}
- currently modelled by modified angular-acceptance curve \cite{icrc17pocam}

- atmospheric neutrinos GeV-TeV range, result from interactions of cosmic ray particles with atmosphere \cite{instrumentation}
- astrophysical acceleration up to PeV scale energies, high-energy phenomena in the universe \cite{instrumentation}
- neutrinos arrive undeflected and unscattered \cite{instrumentation}

- primary scientific objective has been the discovery of astrophyiscal neutrinos \cite{instrumentation} which has been achieved in 2013 \cite{instrumentation,evidence2013}, and identification and characterization of the sources \cite{instrumentation}
- other objectives: indirect detection of dark matter, search exotic matter, studies of neutrino oscillation physics, \cite{instrumentation}
- multi-messenger collaboration with optical, x-ray, gamma-ray, radio, gravitational wave observatories provide multiple windows onto the potential neutrino sources \cite{instrumentation}
- particle energy in the ice 10GeV to 10PeV \cite{instrumentation}



% What is the work?
% Why is it important?
% What is needed to understand the work?
% How will the work be presented?
% Definitions and key terms

% ## What is IceCube
% ## IceCube as Particle Physics Laboratory
% ## IceCube as Deep-space Neutrino Telescope
% ## Improving the Detector Resolution Through Calibration
% ## Can Local Ice Properties be Implemented in Simulations Aiming to Improve Calibration?
