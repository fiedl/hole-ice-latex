%!TEX TS-program = ../make.zsh

\section{Discussion}
\label{sec:discussion}

\subsection{Comparison to Other Hole-Ice-Simulation Methods}
\label{sec:comparison_methods}

\subsubsection{A Priori Modification of Angular-Acception Curves}
\label{sec:a_priori_modification_of_angular_acception_curves}\label{sec:a_priori_curve}

% https://github.com/fiedl/hole-ice-study/issues/10

Modifying the simulated angular-acceptance curve $a(\eta)$ of the optical modules to effectively account for the influence of hole ice on the detection of photons in simulations, is the default approach in \clsim (see sections \ref{sec:a_priori_angular_acceptance} and \ref{sec:hole_ice_approximation}).

\docpar{Information on the nominal and the hole-ice-approximating angular-acceptance curves are provided or linked in \issue{10}.}

The respective angluar-acceptance curves have been obtained using lab measurements and simulations using the \photonics photon-propagation software. \cite{icepaper}\cite{lundberg}\cite{photonics}

Figure \ref{fig:Wee4ahYa} shows three modified angular-acceptance curves that approximate the hole-ice effect in comparison to the optical module's angular acceptance based on the lab measurement. The modified curves H1, H2, and H3 assumed that the entire drill hole would be filled with hole ice, corresponding to a hole-ice-cylinder radius of $30\cm$ and assumed geometric hole-ice-scattering lengths of $\lambda\sca\hi = 100\cm$ (H1), $\lambda\sca\hi = 50\cm$ (H2), and $\lambda\sca\hi = 30\cm$ (H3) respectively.

\begin{figure}[htbp]
  \smallerimage{h2-dima-baselines-jpa14}
  \caption{caption \todo{Produce better image with log, with h0, without dima's model}}
  \label{fig:Wee4ahYa}
\end{figure}

Using modified angular-acceptance curves for the optical modules to approximate the hole-ice effect has the advantage that new parameters gained from calibration studies can be inserted into the existing simulations just by replacing the polynomial parameters used for the angular acceptance of the optical modules.

Also, using modified acceptance curves is faster than simulating the hole ice directly as the many scattering steps that correspond to the photons scattering within the hole ice are not actually simulated and just effectively accounted for using when the algorithm decides whether a photon hit is accepted by an optical module.

This method, however relies on strong assumptions: The lab measurements the acceptance curves are based on, have been performed using manifactured ice rather than south-polar ice. The hole-ice modifications are based purely on simulations and do not use any IceCube calibration information.

Furthermore, this method assumes all optical modules having exactly the same properties regarding positioning and relative orientation. The optical modules are assumed to be perfectly centered within the drill hole as well as in the bubble column.

The H2 acceptance curve, which is used in standard-\clsim simulations by default, is used as reference curve for other angular-acceptance plots in this study.


\subsubsection{Angular-Acceptance Fitting Using Flasher Data}
\label{sec:dimas_model}

% Dima's model
% https://github.com/fiedl/hole-ice-study/issues/102
%
% Flasher unfolding, flasher fits
% \cite{flasherdataderivedicemodels}
%
% d.h. man nimmt nicht Eis-Eigenschaften an, sondern fittet die Winkelakzeptanz direkt an Flasher-Daten
%
% https://wiki.icecube.wisc.edu/index.php/MSU_Forward_Hole_Ice
%
% Another advantage of using Dima's hole ice instead of H2 to extend the parametrization is that
% Dima's hole ice has a somewhat simple formula with a single parameter p as systematic, while the
% H2 model was made via fitting the angular acceptance curve of different simulated hole ices with
% different scattering lengths which means each of the fits turned out somewhat different as seen in
% the figure above and one parametrization would not be as easily transferable to the hole ice at
% different scattering parameters.
%
\chirkin suggests a hole-ice model, often refered to as \textit{Dima's model}, resulting in an angular-acceptance curve, $a\domdima(\eta;p)$, different from the a priori curves for the H1, H2, H3 models.\cite{flasherdataderivedicemodels}

The curve is parameterized using a single parameter $p\in\reals$ that has been determined as $p \in [0.2;0.4]$ using flasher studies. \cite{msuforwardholeice} In this model, the properties $\HH$ of the hole ice, that is to say the hole-ice-cylinder radius $r$ and the scattering length $\lambda\sca\hi$ within the hole-ice are unknown.

\begin{align}
  a\domdima(\eta;p) = & \ 0.34 \cdot (1 + 1.5\,\cos(\eta) - \cos(\eta)^{\sfrac{3}{2}}) \nonumber \\
      & + p \cdot \cos(\eta) \cdot  (\cos(\eta)^2 - 1)^3 \\
      & \ p \in [0.2;0.4] \nonumber
  \label{eq:adima}
\end{align}

Figure \ref{fig:Vohn9Oov} shows this angular-acceptance curve, $a\domdima(\eta;p)$, for the parameters $p=0.2$, $p=0.3$, and $p=0.4$, compared to the a priori angular-acceptance curve $a\domhi(\eta)$ of the H2 model and the angular-acceptance curve of an optical module $a\dom(\eta)$ without hole ice.

\begin{figure}[htbp]
  \smallerimage{h2-vs-dima-vs-h0-102}
  \caption{Comparing the a priori angular-acceptance curve from \cite{icepaper} to Dimas's model \cite{flasherdataderivedicemodels}. Also, the angular acceptance of the optical module (DOM) without any hole ice is shown.}
  \label{fig:Vohn9Oov}
\end{figure}

Notably, the effect of the hole ice on photons approaching the optical module directly from above or below are weaker in this model compared to the H2 approximation curve. However, the overall acceptance in the lower half sphere is higher, and the overall acceptance in the upper half sphere is lower as in the H2 approximation curve.

This approach does not rely on specific assumptions on the properties of the hole ice, but uses flasher calibration data to measure the mean angular-acceptance behaviour of IceCube's optical modules.

Neverthelss, this method relies on the same assumptions regarding the equality of all optical modules: All optical modules are assumed to have the same relative position and orientation with respect to the hole ice.


\subsubsection{Direct Photon Propagation With \ppc}

\chirkin and \rongen have performed simulations with direct propagation through hole ice, similar to this study, but using \ppc for the photon propagation rather than \clsim. \cite{martinspicehddard}\cite{martindardupdate}\cite{pocam}\cite{icrc17pocam}\cite{ppcpaper}

\sourcepar{The source code of \ppc can be found at \url{http://code.icecube.wisc.edu/projects/icecube/browser/IceCube/projects/ppc}.}

Simulating the propagation of photons through the hole ice by simulating the scattering directly makes the assumption obsolete that all optical modules need to be equally positioned and oriented with respect to the hole ice. All optical modules can be individually calibrated using flasher data. \cite{martinspicehddard}

The basic propagation algorithm is the same in \ppc and \clsim. The algorithm for propagating the photons through different media, relies on the same principle of converting randomized numbers of interaction lengths to a geometric distances, but is implemented differently: Ice layers and hole-ice cylinders are hard-coded in \ppc whereas in \clsim's new medium-propagation algorithm, they are treated as generic medium changes (see section \ref{sec:algorithm_b}). Both, \ppc and \clsim, can be run on clusters of graphics processing units.

Hardware requirements and performance of \ppc and \clsim are comparable. Due to micro optimizations, \ppc tends to run slightly faster. \mref{}

\ppc does only support one hole-ice cylinder per string. The main cable of the detector string is not implemented as absorbing object but rather accounted for by using a direction-dependent shadowing factor for incoming photons. \cite{ppcsource}\cite{ppcforhumans}


\subsection{Comparison to Hole-Ice Parameters From Other Studies}
\label{sec:angular_acceptance_comparison}\label{sec:parameter_comparison}

% https://github.com/fiedl/hole-ice-study/issues/104



% https://github.com/fiedl/hole-ice-study/issues/80
%
% Albrecht Karle, Hole Ice Studies with YAG, 1998:
% http://icecube.berkeley.edu/kurt/interstring/hole-ice/yak.html
%
% This YAG laser analysis suggests that the hole ice is described
% best by a geometric scattering length in the range between 50 cm and 100 cm
%
% Hole ice models
% Model 0: hole ice identical to bulk ice
% Model 1: L_scatt = 100 cm (missing in this analysis)
% Model 2: L_scatt = 50 cm
% Model 3: L_scatt = 30 cm
% Model 4: L_scatt = 10 cm
% Scattering length = geometric scattering length on air bubbles.
% Hole diameter = 60 cm.
%
% https://wiki.icecube.wisc.edu/index.php/Hole_ice
%
% These were derived many years ago, and are characterised by the scattering length of the bubble
% column. The H2 model is typically referred to as the "baseline" model, which corresponds to a
% scattering length of 50cm. In simulation, the effects of the H* models are parametrised via angular
% acceptance curves (see above) that determine the probability to accept a photon based on its arrival
% direction (not position) when it intersects with the DOM surface. The derivation of these angular
% acceptance curves assumed that the entire drill hole is full of the scattering centres. However from
% the Sweden camera images we suspect that this hypothesis is incorrect, and the bubbles instead
% concentrate in the centre of the drill hole rather than throughout.
%
\subsubsection{YAG H2 Parameters}
Based on laser measurements, using a Yttrium-Aluminium-Granat (YAG) laser, \authorname{Karle} suggested a number of hole ice models \cite{holeicestudieswithyag}, from which the so called \textit{H2 model} is the dominant hole-ice model in IceCube. This model suggests a geometric scattering length of $50\cm$ for the hole ice, and a hole-ice-cylinder radius of $30\cm$, and is the basis for the modified angular-acception curve with hole-ice approximation described in section \ref{sec:a_priori_modification_of_angular_acception_curves}.

Using the new direct photon propagation though hole ice with \clsim, one can simulate the propagation through hole ice with parameters suggested by the H2 model, scan the angular acceptance of an optical module within the simulation, and compare the result to the existing a priori angular-acceptance curve (section \ref{sec:a_priori_curve}) as well as the angular-acceptance curve suggested by Dima's model (section \ref{sec:dimas_model}).

\docpar{This angular-acceptance simulation using the H2 hole-ice parameters is documented in \issue{80}.}

Figure \ref{fig:xaeg2Mee} shows the result of this angular-acceptance simulation in comparison to the a priori angular-acception curve with hole-ice approximation \cite{icepaper} and Dima's model \cite{flasherdataderivedicemodels}.

\begin{figure}[htbp]
  \smallerimage{angular-acceptance-karle-h2-xaeg2Mee}
  \caption{Angular-acceptance simulation with hole-ice parameters from the so called \textit{H2 model} \cite{holeicestudieswithyag}, which describes the hole ice as cylinder of $30\cm$ radius filling the entire drill hole, with a geometric scattering length of $50\cm$, corresponding to an effective scattering length of $\lambda\hi\esca = 8.33\m$, using the new medium-propagation algorithm (section \ref{sec:algorithm_b}) with direct detection as acception criterion and plane waves as photon sources. In comparison, the a priori angular-acceptance curve from \cite{icepaper} and the angular-acceptance curve of Dima's model \cite{flasherdataderivedicemodels} are shown.}
  \label{fig:xaeg2Mee}
\end{figure}

For photons approaching the optical module from below (right-hand side of the plot), the simulation rather follows Dima's model, but records more detected photons in total. For photon approaching from above (left-hand side of the plot), the form of the simulation curve is similar to the a priori curve, in particluar with respect to the plateau on the left hand side, but, again, the simulation records more photons in total.


\subsubsection{\noun{DARD} Parameters}

In the \noun{DARD} study (\noun{Data Aquisition for a flasheR DOM}), LED measurements are used to determine the hole-ice properties. The different flasher LEDs of an optical module are turned on in sequence, and the light detected by the same DOM is measured. The process is then compared to detailed \noun{Geant4} simulations for different hole-ice parameters in order to find the optimal hole-ice parameters to explain the data. \cite{martindardupdate}\cite{martinspicehddard}

The \noun{DARD} study assumes a fixed hole-ice radius of $r=30\cm$. The best fit value for the geometric scattering length of the hole ice has been determined as $\lambda\hi\sca = 10\cm$, which is considerably lower than other estimates. \cite{martindardupdate}

Using the new direct photon propagation though hole ice with \clsim, one can simulate the propagation through hole ice with parameters suggested by \noun{DARD}, scan the angular acceptance of an optical module within the simulation, and compare the result to the existing a priori angular-acceptance curve as well as the angular-acceptance curve suggested by Dima's model.

\docpar{This angular-acceptance simulation for the hole-ice parameters suggested by the \noun{DARD} study, is documented in \issue{105}.}

Figure \ref{fig:eePai1sh} shows the result of this angular-acceptance simulation in comparison to the a priori angular-acception curve with hole-ice approximation \cite{icepaper} and Dima's model \cite{flasherdataderivedicemodels}.

\begin{figure}[htbp]
  \smallerimage{angular-acceptance-dard-eePai1sh}
  \caption{Angular-acceptance simulation with hole-ice parameters from the \noun{DARD} study \cite{martindardupdate}, which suggest a hole-ice cylinder of $30\cm$ radius and a geometric scattering length of $\lambda\sca\hi = 10\cm$, corresponding to an effective scattering length of $\lambda\hi\esca = 1.67\m$, using the new medium-propagation algorithm (section \ref{sec:algorithm_b}) with direct detection as acception criterion and plane waves as photon sources. In comparison, the a priori angular-acceptance curve from \cite{icepaper} and the angular-acceptance curve of Dima's model \cite{flasherdataderivedicemodels} are shown.}
  \label{fig:eePai1sh}
\end{figure}

For photons approaching the optical module from below (right-hand side of the plot), the simulation curve rather follows the a priori acceptance curve, but shows a less sharp hole-ice drop-off effect. For $\cos\eta < 0.5$, that is to say for higher angles, the simulation registers relatively more photons, both in comparison to the a priori curve and to Dima's model.


\subsubsection{\noun{SpiceHD} Parameters}
The \noun{SpiceHD} study (\noun{South Pole ICE model with Hole ice fitting and Direct detection}) uses flasher data and direct photon propagation simulations with \ppc to determine the parameters of the hole ice. In this 7-string flasher study, LED flashes are sent from a central optical module and the numbers of photon hits are count at the optical modules of the six surrounding strings.

Comparing the flasher data to simulations with different hole-ice parameters, that is to say scanning over the hole-ice-cylinder radius and the hole-ice scattering length, a range of hole-ice parameters is found to be suitable (see \mref{flasher scan comparison section}). When assuming a compatible hole-ice column radius as observed by camera images \mref{cite swedish camera paper}, the best fit hole-ice parameters have been determined as a hole-ice-cylider radius of $60\,\%$ of the radius of an optical module, $r=0.6\,r\dom$, and the hole-ice effective scattering length $\lambda\esca\hi = 14\cm$, which corresponds to a geometric scattering length of $\lambda\sca\hi = 0.84\cm$, which is even smaller than the geometric scattering length of $10\cm$ suggested by \noun{DARD}. \cite{martinspicehddard}

Using the new direct photon propagation though hole ice with \clsim, one can simulate the propagation through hole ice with parameters suggested by \noun{SpiceHD}, scan the angular acceptance of an optical module within the simulation, and compare the result to the existing a priori angular-acceptance curve as well as the angular-acceptance curve suggested by Dima's model.

\docpar{This angular-acceptance simulation for the hole-ice parameters suggested by the \noun{SpiceHD} study, is documented in \issue{87}.}

Figure \ref{fig:ku3Zie8z} shows the result of this angular-acceptance simulation in comparison to the a priori angular-acception curve with hole-ice approximation \cite{icepaper} and Dima's model \cite{flasherdataderivedicemodels}.

\begin{figure}[htbp]
  \smallerimage{angular-acceptance-splicehd-ku3Zie8z}
  \caption{Angular-acceptance simulation with hole-ice parameters suggested by the \noun{SpiceHD} study \cite{martinspicehddard}: The dominant hole ice is assumed as a bubble colum with radius $r = 0.6\,r\dom$ where $r\dom$ is the radius of the optical module. The bubble column is thinner than the optical module. The hole-ice effective scattering length is $\lambda\esca\hi = 14\cm$. The simulation uses the new medium-propagation algorithm (section \ref{sec:algorithm_b}) with direct detection as acception criterion and plane waves as photon sources. The simulation results are shown in comparison with the a priori angular-acceptance curve from \cite{icepaper} and the angular-acceptance curve of Dima's model \cite{flasherdataderivedicemodels}.}
  \label{fig:ku3Zie8z}
\end{figure}

For photons that approach the optical module from below (right-hand side of the plot), the simulation shows a hole-ice effect between the a priori curve and Dima's model. For photons approaching from above, the simulation shows a plateau. For $\cos\eta < 0.5$, the simulation shows more photon hits as both, the a priori curve and Dima's model.




- POCAM munich paper, S.9, martin tests his direct detection
  martin simulates entire drill hole filled (r = 30cm) with esca=125cm, sca = 7.5cm
  which maches the ppc spline model and the h2 curve

- POCAM, S. 10, r = 0.6r_dom, esca = 14cm, between Dima and H2




\paragraph{Simulations for Hole-Ice Parameters From Other Studies}
- results from martin
- ice paper 2013: #80 -> check this issue for updates on why the do not match the reference curve
- further investigations are required to find out why the ice paper parameters do not match the ice paper a priori curve in this simuation setup
- first guess would be some systematic error in simulations, but cross checks look good
- maybe some systematic error in angular-acceptance-simulation setup

- also scan over parameters and compare agreement of a priori curve and simulation curve -> next section