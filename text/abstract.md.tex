%!TEX TS-program = ../make.zsh

\cleardoublepage
\thispagestyle{empty}
\begin{abstract}

\icecube is a neutrino observatory at Earth's South Pole that uses glacial ice as detector medium where secondary particles from neutrino interactions produce Cherenkov light, which then is detected by an array of photo detectors deployed within the ice. Aiming to improve detector precision, this study introduces a method to simulate the propagation of light through the \textit{hole ice} in proximity to the detector modules, introducing several new calibration parameters.

The validity of the method is supported by a series of statistical cross checks, and by comparison to  measurement and simulation results from other studies.

Calibration data indicates an asymmetric shielding of the detector modules that cannot be accounted for by shadowing cables, but can be explained by simulating hole ice with suitable position, size and scattering length.

Additional simulation time required by direct hole-ice propagation suggests that high-statistics studies should continue to use hole-ice approximation techniques, which, however, need to be adjusted according to direct simulation results. For low-statistics studies, direct hole-ice simulation can be used to reduce systematic uncertainties, which, however need to be quantified by follow-up studies.

\end{abstract}