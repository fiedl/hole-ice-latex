%!TEX TS-program = ../make.zsh

\cleardoublepage
\thispagestyle{empty}
\begin{abstract}

\icecube is a neutrino observatory at Earth's South Pole that uses glacial ice as detector medium. Secondary particles from neutrino interactions produce Cherenkov light, which is detected by an array of photo detectors deployed within the ice. In distinction from the glacial bulk ice, hole ice is the refrozen water in the drill holes that were needed to deploy the detector modules, and is expected to have different optical properties than the bulk ice.
Aiming to improve detector precision, this study introduces a new method to simulate the propagation of light through the hole ice, introducing several new calibration parameters. The validity of the method is supported by a series of statistical cross checks, and by comparison to measurement and simulation results from other calibration studies.
Calibration data indicates a strongly asymmetric shielding of the detector modules. Preliminary results hint that this cannot be accounted for by shadowing cables, but can be explained by simulating hole ice with a suitable scattering length, size, and position relative to the detector modules.
The direct hole-ice simulation will be integrated in \icecube's simulation framework. Low-statistics studies can use the direct simulation to reduce systematic uncertainties. As direct hole-ice propagation requires additional simulation time, high-statistics studies should continue to use hole-ice approximation techniques, which, however, need to be adjusted according to direct-simulation results.

\end{abstract}