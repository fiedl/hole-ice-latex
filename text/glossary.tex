%!TEX TS-program = ../make.zsh

\subsection{List of Abbreviations}

\begin{tabelle}{lL}
  \clsim & OpenCL-based photon-tracking simulation using a (source-based) ray tracing algorithm modeling scattering and absorption of light in the deep glacial ice at the South Pole or Mediterranean sea water. See \cite{clsimreadme, clsimsource}. \\
  DOM & Digital Optical Module, primary instrumentation and detection unit in \icecube, see section \ref{sec:doms}. \\
  H0 & Hypothesis stating that no hole ice exists in the \icecube detector. See \cite{yag}. \\
  H1 & Hypothesis stating that hole ice exists in the \icecube detector with a hole-ice radius of $30\cm$, filling the entire drill hole, and a geometric hole-ice scattering length of $100\cm$. See \cite{yag}. \\
  H2 & Hypothesis stating that hole ice exists in the \icecube detector with a hole-ice radius of $30\cm$, filling the entire drill hole, and a geometric hole-ice scattering length of $50\cm$. See \cite{yag}. \\
  H3 & Hypothesis stating that hole ice exists in the \icecube detector with a hole-ice radius of $30\cm$, filling the entire drill hole, and a geometric hole-ice scattering length of $30\cm$. See \cite{yag}. \\
  H4 & Hypothesis stating that hole ice exists in the \icecube detector with a hole-ice radius of $30\cm$, filling the entire drill hole, and a geometric hole-ice scattering length of $10\cm$. See \cite{yag}. \\
  LED & Light-emitting diode, part of \icecube's flasher calibration system, see section \ref{sec:flasher}. \\
  \ppc & photon propagation code. A photon-propagation simulation software. See \cite{ppcpaper, ppcsource, ppcforhumans}. \\
  PMT & Photo Multiplier Tube, instrument to detect light, component of \icecube's optical modules, see section \ref{sec:doms}. \\
  YAG & Yttrium-Aluminium-Garnet, synthetic material, used as laser medium in ice studies, see section \ref{sec:yag_h2_parameters}. \\
\end{tabelle}

\subsection{List of Quantities}\nopagebreak

\begin{tabelle}{lL}
  $\lambda\sca$ & geometric scattering length, in general or within the bulk ice, see section \ref{sec:scattering}. \\
  $\lambda\sca\hi$ & geometric scattering length within the hole ice. \\
  $\lambda\esca$ & effective scattering length, in general or within the bulk ice, $\lambda\esca = \frac{\lambda\sca}{1 - \meancostheta}$, see section \ref{sec:scattering}. \\
  $\lambda\esca\hi$ & effective scattering length within the hole ice, $\lambda\esca\hi = \frac{\lambda\sca\hi}{1 - \meancostheta}$. \\
  $\lambda\abs$ & absorption length, in general or within the bulk ice, see section \ref{sec:scattering}. \\
  $\lambda\abs\hi$ & absorption length within the hole ice. \\
  $r$ & radius of the hole-ice cylinder, see section \ref{sec:hole_ice}. \\
  $r\dom$ & radius of \icecube's digital optical modules (DOMs), see section \ref{sec:doms}. \\
\end{tabelle}


\subsection{List of Units}

\paragraph{Units of Length} \mbox{}

\begin{tabelle}{lL}
  $\m$ & meter, base unit of length in the international system of units \\
  $\cm$ & centimeter, $10^{-2}\m$ \\
  $\mm$ & millimeter, $10^{-3}\m$ \\
  $\nm$ & nanometer, $10^{-9}\m$ \\
\end{tabelle}

\paragraph{Units of Time} \mbox{}

\begin{tabelle}{lL}
  $\unit{s}$ & second, base unit of time in the international system of units \\
  $\ns$ & nanosecond, $10^{-9}\unit{s}$ \\
\end{tabelle}

\newpage
\paragraph{Units of Energy} \mbox{}

\begin{tabelle}{lL}
  $\unit{eV}$ & electron volt, unit of energy, $1\unit{eV} = 1.6021766208 \cdot 10^{-19}$ joules, which is the basic unit of energy in the international system of units \\
  $\unit{MeV}$ & mega electron volt, $10^{6}\unit{eV}$ \\
  $\GeV$ & giga electron volt, $10^{9}\unit{eV}$ \\
  $\TeV$ & tera electron volt, $10^{12}\unit{eV}$ \\
  $\PeV$ & peta electron volt, $10^{15}\unit{eV}$ \\
\end{tabelle}
\vfill

\subsection{List of Mathematical Symbols}

\begin{tabelle}{lL}
  $:$ & property operator. $A:B$ means that object $A$ has the property $B$. The associativity is focused on $A$, such that \enquote{The radius of the cylinder is $r:B$} means \enquote{The radius of the cylinder is $r$ and $r$ has the property $B$.} \\
  $=$ & equality. $A=B$ means that the quantities $A$ and $B$ have the same value. \\
  $:=$ & short cut for property operator in conjunction with the equality operator. $A:=B$ means $A:A=B$. \\
  $\approx$ & approximate equality. $A \approx B$ means that the quantities $A$ and $B$ have approximately the same value, where the precision of the approximation is given by the context of the statement. \\
  $\equiv$ & definition equality operator. $A \equiv B$ means that the symbols $A$ and $B$ represent the same object. \\
  $\sum$ & sum. $\sum_{i=1}^n a_i$ means $a_1 + a_2 + \dots + a_n$ \\
  $\prod$ & product. $\prod_{i=1}^n a_i$ means $a_1 \cdot a_2 \cdot \dots \cdot a_n$ \\
  $\pi$ & circle constant, representing the ratio of a circle's circumference to its diameter, $\pi \approx 3.14159$ \\
  $\{ \}$ & set. $\{ A, B \}$ represents the set of the objects $A$ and $B$. \\
  $\naturals$ & set of all natural numbers \\
  $\reals$ & set of all real numbers \\
  $\reals^+$ & set of all positive real numbers \\
  $\reals^+_0$ & set of all positive real numbers and zero \\
  $\ket{A}$ & quantum mechanical bra-ket notation for the quantum state of $A$ as abstract column vector \\
\end{tabelle}
