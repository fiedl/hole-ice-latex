\section{Conclusion}
\label{sec:conclusion}

This study presented a new method to simulate the propagation of photons
through the hole ice around the detector strings of the
\icecube neutrino observatory as an extension of the standard
\clsim photon-propagation-simulation software.

Two algorithms were introduced to accomplish this task. As a first
approach, an algorithm was developed that accounts for the different
optical properties of the hole ice by adding subsequent corrections to
the calculations of the standard-\clsim algorithm. The changes to the
well-tested standard-\clsim codebase are kept minimal. The algorithm
requires the definition of the hole-ice properties to be relative to the
properties of the surrounding ice, which, however, does not allow to
implement more current hole-ice models. Therefore, a second algorithm
was devised that handles the transition from bulk ice to hole ice and
vice versa the same way it handles other medium transitions like the
propagation through ice layers, and allows to simulate hole ice with
absolute scattering and absorption lengths. Because this approach is
more generic, it also allows more complex hole-ice scenarios where a
photon crosses more than one hole-ice cylinder between two scatters.
This approach, however, required to rewrite the previous
\clsim media-propagation code. The layer tilt and ice anisotropy are
left out in the second algorithm for now, but can be added in the
future. The validity of the algorithms is supported by unit tests and a
series of statistical cross checks. \clsim hole-ice simulation results
were found to agree with simulation results of \ppc, which is a separate
\icecube simulation software that also supports direct hole-ice
propagation, but uses a different approach to implementing the medium
transitions. The performance difference per scattering step of the
standard-\clsim media propagation and the new hole-ice algorithms is
negligible within the statistical uncertainties. Simulating the direct
photon propagation through hole ice with small scattering lengths,
however, adds more scattering steps to the simulation, resulting in a
longer total simulation time. The performance of large-scale simulations
can still be improved by applying GPU-programming and memory
optimizations.

In general, the new method allows to simulate the propagation through
hole-ice cylinders with different absorption lengths, scattering
lengths, and radii. The optical modules can be positioned with
individual displacements relative to the hole ice. The method supports
the nesting of several hole-ice cylinders of different properties and
the simulation of light-absorbing cables. Optical modules support direct
detection and thereby can accept photon hits based on whether the photon
hit the sensitive area of the optical module rather than based on the
impact angle, as done in standard \clsim. These features allow to fit
additional calibration parameters such as the positions, sizes, and
scattering lengths of the individual hole-ice columns in the
\icecube detector.

Direct-photon-propagation simulations indicate that light propagation
through the \icecube detector on large scales is mostly unaffected by
the hole ice. Each photon, however, that is eventually detected by the
optical modules, and every photon that is emitted by the calibration
LEDs, needs to propagate through the hole ice and is affected by the
properties of the hole ice. For sufficiently sized hole-ice columns with
small scattering lengths, the optical modules are effectively shielded
by the hole ice. A fraction of the photons is absorbed during the random
walk through the hole ice, or effectively reflected by the hole ice.
Evaluating calibration data indicates a strongly asymmetric shielding of
the detector modules. Preliminary flasher simulations with direct photon
propagation hint that this cannot be accounted for by the shadow of
cables, but can be explained by simulating hole ice with a suitable
scattering length, size and position relative to the detector modules.

The new simulation method will be integrated into \icecube's simulation
framework. Low-statistics studies can use the new simulation method to
propagate all photons without hole-ice approximations in order to reduce
systematic uncertainties. As the direct hole-ice propagation requires
additional simulation time, high-statistics studies should continue to
use approximation techniques, which modify the angular-acceptance
behavior of the simulated optical modules to effectively account for the
average effect of the propagation through hole-ice columns. Comparing
the current standard hole-ice approximation to direct simulations, both
methods were found to disagree. Assuming the same hole-ice properties,
the new simulations indicate a less pronounced hole-ice effect than the
approximation curves. As recent calibration studies contradict each
other regarding the strength of the hole-ice properties, there is no
clear indication towards one of these models. With the new simulation
tools, however, more detailed hole-ice-simulation studies can be
performed aiming to find a hole-ice description that allows to reproduce
all calibration observations. After that, the approximation curves can
be updated accordingly to match the new hole-ice model.
