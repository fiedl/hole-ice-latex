%!TEX root = ../diplomarbeit.tex

\subsection{Source Code of Algorithm A}
\label{sec:algorithm_a_source}

This is the source code of the first propagation algorithm through different media, described in section \ref{sec:algorithm_a}.

\sourcepar{The source code can also be found online at \url{https://github.com/fiedl/clsim/tree/sf/hole-ice-2017/resources/kernels/lib}.}

\begin{ccode}
// https://github.com/fiedl/clsim/blob/sf/hole-ice-2017/resources/kernels/lib/hole_ice/hole_ice.c

#ifndef HOLE_ICE_C
#define HOLE_ICE_C

#include "hole_ice.h"
#include "../intersection/intersection.c"

inline bool is_between_zero_and_one(floating_t a) {
  return ((!my_is_nan(a)) && (a > 0.0) && (a < 1.0));
}

inline bool not_between_zero_and_one(floating_t a) {
  return (! is_between_zero_and_one(a));
}

inline unsigned int number_of_medium_changes(HoleIceProblemParameters_t p)
{
  // These are ordered by their frequency of occurrance in order
  // to optimize for performance.
  if (not_between_zero_and_one(p.entry_point_ratio) && not_between_zero_and_one(p.termination_point_ratio)) return 0;
  if (not_between_zero_and_one(p.entry_point_ratio) || not_between_zero_and_one(p.termination_point_ratio)) return 1;
  if (p.entry_point_ratio == p.termination_point_ratio) return 0; // tangent
  return 2;
}

inline floating_t hole_ice_distance_correction(HoleIceProblemParameters_t p)
{
  // Depending on the fraction of the distance the photon is traveling
  // within the hole ice, there are six cases to consider.
  //
  // N  denotes the number of intersections.
  // H  means that the trajectory starts within hole ice.
  // !H means that the trajectory starts outside of the hole ice.
  //
  // Case 1: The trajectory is completely outside of the hole ice (!H, N=0).
  // Case 2: The trajectory is completely within the hole ice (H, N=0).
  // Case 3: The trajectory begins outside, but ends inside the hole ice (!H, N=1).
  // Case 4: The trajectory begins inside, but ends outside the hole ice (H, N=1).
  // Case 5: The trajectory starts and ends outside, but passes through the hole ice (!H, N=2).
  // Case 6: The trajectory begins within one hole-ice cylinder, passes through
  //           normal ice and ends in another hole-ice cylinder (H, N=2).
  //
  // For further information, please have look at:
  // https://github.com/fiedl/clsim/tree/sf/master/resources/kernels/lib/hole_ice

  const unsigned int num_of_medium_changes = number_of_medium_changes(p);

  // Case 1: The trajectory is completely outside of the hole ice.
  // Thus, needs no correction.
  if ((num_of_medium_changes == 0) && !p.starts_within_hole_ice) {
    return 0;
  }

  const floating_t ac = p.distance * p.termination_point_ratio;

  if (p.starts_within_hole_ice) {

    // Case 4: The trajectory begins inside, but ends outside the hole ice.
    if (p.interaction_length_factor * p.distance > ac) {
      return (1.0 - 1.0 / p.interaction_length_factor) * ac;

    // Case 2: The trajectory is completely within the hole ice.
    } else {
      return (p.interaction_length_factor - 1.0) * p.distance;
    }

  } else {

    const floating_t yb = p.distance * (1.0 - p.entry_point_ratio);
    floating_t yc = p.distance * (p.termination_point_ratio - p.entry_point_ratio);

    if (yc < 0) {
      printf("WARNING: YC SHOULD NOT BE NEGATIVE, BUT YC=%f\n", yc);
      yc = 0;
    }

    // Case 5: The trajectory starts and ends outside, but passes through the hole ice.
    if (p.interaction_length_factor * yb > yc) {
      return (1.0 - 1.0 / p.interaction_length_factor) * yc;

    // Case 3
    } else {
      return (p.interaction_length_factor - 1.0) * p.distance * (1.0 - p.entry_point_ratio);
    }
  }

#ifdef PRINTF_ENABLED
  printf("WARNING: UNHANDLED INTERSECTION CASE. This point should not be reached.");
#endif

  return my_nan();
}

inline floating_t hole_ice_distance_correction_for_intersection_problem(floating_t distance, floating_t interaction_length_factor, IntersectionProblemParameters_t p)
{
  calculate_intersections(&p);
  HoleIceProblemParameters_t hip = {
    distance,
    interaction_length_factor,
    intersection_s1(p), // entry_point_ratio
    intersection_s2(p), // termination_point_ratio
    intersecting_trajectory_starts_inside(p) // starts_within_hole_ice
  };
  return hole_ice_distance_correction(hip);
}

#ifdef HOLE_ICE_TEST_H
inline floating_t apply_hole_ice_correction(floating4_t photonPosAndTime, floating4_t photonDirAndWlen, unsigned int numberOfCylinders, floating4_t *cylinderPositionsAndRadii, floating_t holeIceScatteringLengthFactor, floating_t holeIceAbsorptionLengthFactor, floating_t *distancePropagated, floating_t *distanceToAbsorption)
#endif
#ifndef HOLE_ICE_TEST_H
inline floating_t apply_hole_ice_correction(floating4_t photonPosAndTime, floating4_t photonDirAndWlen, unsigned int numberOfCylinders, __constant floating4_t *cylinderPositionsAndRadii, floating_t holeIceScatteringLengthFactor, floating_t holeIceAbsorptionLengthFactor, floating_t *distancePropagated, floating_t *distanceToAbsorption)
#endif
{

  if (!(my_is_nan(photonPosAndTime.x) || my_is_nan(*distancePropagated))) {

    // Find out which cylinders are in range in a separate loop
    // in order to improve parallelism and thereby performance.
    //
    // See: https://github.com/fiedl/hole-ice-study/issues/30
    //
    #ifdef NUMBER_OF_CYLINDERS
      // When running this on OpenCL, defining arrays using a constant
      // as array size is not possible. Therefore, we need to use a
      // pre-processor makro here.
      //
      // See: https://github.com/fiedl/hole-ice-study/issues/38
      //
      int indices_of_cylinders_in_range[NUMBER_OF_CYLINDERS];
    #else
      int indices_of_cylinders_in_range[numberOfCylinders];
    #endif
    {
      unsigned int j = 0;
      for (unsigned int i = 0; i < numberOfCylinders; i++) {
        indices_of_cylinders_in_range[i] = -1;
      }
      for (unsigned int i = 0; i < numberOfCylinders; i++) {
        if (sqr(photonPosAndTime.x - cylinderPositionsAndRadii[i].x) +
            sqr(photonPosAndTime.y - cylinderPositionsAndRadii[i].y) <=
            sqr(*distancePropagated + cylinderPositionsAndRadii[i].w /* radius */))
        {

          // If the cylinder has a z-range check if we consider that cylinder
          // to be in range. https://github.com/fiedl/hole-ice-study/issues/34
          //
          if ((cylinderPositionsAndRadii[i].z == 0) || ((cylinderPositionsAndRadii[i].z != 0) && !(((photonPosAndTime.z < cylinderPositionsAndRadii[i].z - 0.5) && (photonPosAndTime.z + *distancePropagated * photonDirAndWlen.z < cylinderPositionsAndRadii[i].z - 0.5)) || ((photonPosAndTime.z > cylinderPositionsAndRadii[i].z + 0.5) && (photonPosAndTime.z + *distancePropagated * photonDirAndWlen.z > cylinderPositionsAndRadii[i].z + 0.5)))))
          {
            indices_of_cylinders_in_range[j] = i;
            j += 1;
          }
        }
      }
    }

    // Now loop over all cylinders in range and calculate corrections
    // for `distancePropagated` and `distanceToAbsorption`.
    //
    for (unsigned int j = 0; j < numberOfCylinders; j++) {
      const int i = indices_of_cylinders_in_range[j];
      if (i == -1) {
        break;
      } else {

        IntersectionProblemParameters_t p = {

          // Input values
          photonPosAndTime.x,
          photonPosAndTime.y,
          cylinderPositionsAndRadii[i].x,
          cylinderPositionsAndRadii[i].y,
          cylinderPositionsAndRadii[i].w, // radius
          photonDirAndWlen,
          *distancePropagated,

          // Output values (will be calculated)
          0, // discriminant
          0, // s1
          0  // s2

        };

        calculate_intersections(&p);

        // Are intersection points possible?
        if (intersection_discriminant(p) > 0) {

          const floating_t scatteringEntryPointRatio = intersection_s1(p);
          const floating_t scatterintTerminationPointRatio = intersection_s2(p);

          HoleIceProblemParameters_t scatteringCorrectionParameters = {
            *distancePropagated,
            holeIceScatteringLengthFactor,
            scatteringEntryPointRatio,
            scatterintTerminationPointRatio,
            intersecting_trajectory_starts_inside(p)
          };

          const floating_t scaCorrection = hole_ice_distance_correction(scatteringCorrectionParameters);
          *distancePropagated += scaCorrection;

          // For the absorption, there are special cases where the photon is scattered before
          // reaching either the first or the second absorption intersection point.
          floating_t absorptionEntryPointRatio;
          floating_t absorptionTerminationPointRatio;
          floating_t absCorrection = 0.0;
          if (!(not_between_zero_and_one(scatteringCorrectionParameters.entry_point_ratio) && !scatteringCorrectionParameters.starts_within_hole_ice)) {
            // The photon reaches the hole ice, i.e. the absorption correction
            // needs to be calculated.
            p.distance = *distanceToAbsorption;
            calculate_intersections(&p);

            // If the photon is scattered away before reaching the far and of
            // the hole ice, the affected trajectory is limited by the
            // point where the photon is scattered away.
            absorptionEntryPointRatio = intersection_s1(p);
            absorptionTerminationPointRatio = min(
              *distancePropagated / *distanceToAbsorption,
              intersection_s2(p)
            );

            HoleIceProblemParameters_t absorptionCorrectionParameters = {
              *distanceToAbsorption,
              holeIceAbsorptionLengthFactor,
              absorptionEntryPointRatio,
              absorptionTerminationPointRatio,
              intersecting_trajectory_starts_inside(p),
            };

            absCorrection = hole_ice_distance_correction(absorptionCorrectionParameters);

          }
          *distanceToAbsorption += absCorrection;

        }

      }
    }
  }

}

#endif
\end{ccode}

\begin{ccode}
// https://github.com/fiedl/clsim/blob/sf/hole-ice-2017/resources/kernels/lib/hole_ice/hole_ice.h

#ifndef HOLE_ICE_H
#define HOLE_ICE_H

typedef struct HoleIceProblemParameters {
  floating_t distance;
  floating_t interaction_length_factor;
  floating_t entry_point_ratio;
  floating_t termination_point_ratio;
  bool starts_within_hole_ice;
} HoleIceProblemParameters_t;

#endif
\end{ccode}

\begin{ccode}
// https://github.com/fiedl/clsim/blob/sf/hole-ice-2017/resources/kernels/lib/intersection/intersection.c

#include "intersection.h"

inline void calculate_intersections(IntersectionProblemParameters_t *p)
{
  // Step 1
  const floating4_t vector_AM = {p->mx - p->ax, p->my - p->ay, 0.0, 0.0};
  const floating_t xy_projection_factor = my_sqrt(1 - sqr(p->direction.z));
  const floating_t length_AMprime = dot(vector_AM, p->direction) / xy_projection_factor;

  // Step 2
  p->discriminant = sqr(p->r) - dot(vector_AM, vector_AM) + sqr(length_AMprime);

  // Step 3
  const floating_t length_XMprime = my_sqrt(p->discriminant);

  // Step 4
  const floating_t length_AX1 = length_AMprime - length_XMprime;
  const floating_t length_AX2 = length_AMprime + length_XMprime;
  p->s1 = length_AX1 / p->distance / xy_projection_factor;
  p->s2 = length_AX2 / p->distance / xy_projection_factor;
}

inline floating_t intersection_s1(IntersectionProblemParameters_t p)
{
  return p.s1;
}

inline floating_t intersection_s2(IntersectionProblemParameters_t p)
{
  return p.s2;
}

inline floating_t intersection_discriminant(IntersectionProblemParameters_t p)
{
  return p.discriminant;
}

inline floating_t intersection_x1(IntersectionProblemParameters_t p)
{
  if ((p.s1 > 0) && (p.s1 < 1))
    return p.ax + p.direction.x * p.distance * p.s1;
  else
    return my_nan();
}

inline floating_t intersection_x2(IntersectionProblemParameters_t p)
{
  if ((p.s2 > 0) && (p.s2 < 1))
    return p.ax + p.direction.x * p.distance * p.s2;
  else
    return my_nan();
}

inline floating_t intersection_y1(IntersectionProblemParameters_t p)
{
  if ((p.s1 > 0) && (p.s1 < 1))
    return p.ay + p.direction.y * p.distance * p.s1;
  else
    return my_nan();
}

inline floating_t intersection_y2(IntersectionProblemParameters_t p)
{
  if ((p.s2 > 0) && (p.s2 < 1))
    return p.ay + p.direction.y * p.distance * p.s2;
  else
    return my_nan();
}

inline bool intersecting_trajectory_starts_inside(IntersectionProblemParameters_t p)
{
  return (intersection_s1(p) <= 0) &&
      (intersection_s2(p) > 0) &&
      (intersection_discriminant(p) > 0);
}

inline bool intersecting_trajectory_starts_outside(IntersectionProblemParameters_t p)
{
  return ( ! intersecting_trajectory_starts_inside(p));
}

inline bool intersecting_trajectory_ends_inside(IntersectionProblemParameters_t p)
{
  return (intersection_s1(p) < 1) &&
      (intersection_s2(p) >= 1) &&
      (intersection_discriminant(p) > 0);
}
\end{ccode}

\begin{ccode}
// https://github.com/fiedl/clsim/blob/sf/hole-ice-2017/resources/kernels/lib/intersection/intersection.h

#ifndef INTERSECTION_H
#define INTERSECTION_H

typedef struct IntersectionProblemParameters {

  // Input values
  //
  floating_t ax;
  floating_t ay;
  floating_t mx;
  floating_t my;
  floating_t r;
  floating4_t direction;
  floating_t distance;

  // Output values, which will be calculated in
  // `calculate_intersections()`.
  //
  floating_t discriminant;
  floating_t s1;
  floating_t s2;

} IntersectionProblemParameters_t;

#endif
\end{ccode}

